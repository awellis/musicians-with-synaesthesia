% Options for packages loaded elsewhere
\PassOptionsToPackage{unicode}{hyperref}
\PassOptionsToPackage{hyphens}{url}
%
\documentclass[
  man,floatsintext]{apa7}
\usepackage{amsmath,amssymb}
\usepackage{iftex}
\ifPDFTeX
  \usepackage[T1]{fontenc}
  \usepackage[utf8]{inputenc}
  \usepackage{textcomp} % provide euro and other symbols
\else % if luatex or xetex
  \usepackage{unicode-math} % this also loads fontspec
  \defaultfontfeatures{Scale=MatchLowercase}
  \defaultfontfeatures[\rmfamily]{Ligatures=TeX,Scale=1}
\fi
\usepackage{lmodern}
\ifPDFTeX\else
  % xetex/luatex font selection
\fi
% Use upquote if available, for straight quotes in verbatim environments
\IfFileExists{upquote.sty}{\usepackage{upquote}}{}
\IfFileExists{microtype.sty}{% use microtype if available
  \usepackage[]{microtype}
  \UseMicrotypeSet[protrusion]{basicmath} % disable protrusion for tt fonts
}{}
\makeatletter
\@ifundefined{KOMAClassName}{% if non-KOMA class
  \IfFileExists{parskip.sty}{%
    \usepackage{parskip}
  }{% else
    \setlength{\parindent}{0pt}
    \setlength{\parskip}{6pt plus 2pt minus 1pt}}
}{% if KOMA class
  \KOMAoptions{parskip=half}}
\makeatother
\usepackage{xcolor}
\usepackage{color}
\usepackage{fancyvrb}
\newcommand{\VerbBar}{|}
\newcommand{\VERB}{\Verb[commandchars=\\\{\}]}
\DefineVerbatimEnvironment{Highlighting}{Verbatim}{commandchars=\\\{\}}
% Add ',fontsize=\small' for more characters per line
\usepackage{framed}
\definecolor{shadecolor}{RGB}{248,248,248}
\newenvironment{Shaded}{\begin{snugshade}}{\end{snugshade}}
\newcommand{\AlertTok}[1]{\textcolor[rgb]{0.94,0.16,0.16}{#1}}
\newcommand{\AnnotationTok}[1]{\textcolor[rgb]{0.56,0.35,0.01}{\textbf{\textit{#1}}}}
\newcommand{\AttributeTok}[1]{\textcolor[rgb]{0.77,0.63,0.00}{#1}}
\newcommand{\BaseNTok}[1]{\textcolor[rgb]{0.00,0.00,0.81}{#1}}
\newcommand{\BuiltInTok}[1]{#1}
\newcommand{\CharTok}[1]{\textcolor[rgb]{0.31,0.60,0.02}{#1}}
\newcommand{\CommentTok}[1]{\textcolor[rgb]{0.56,0.35,0.01}{\textit{#1}}}
\newcommand{\CommentVarTok}[1]{\textcolor[rgb]{0.56,0.35,0.01}{\textbf{\textit{#1}}}}
\newcommand{\ConstantTok}[1]{\textcolor[rgb]{0.00,0.00,0.00}{#1}}
\newcommand{\ControlFlowTok}[1]{\textcolor[rgb]{0.13,0.29,0.53}{\textbf{#1}}}
\newcommand{\DataTypeTok}[1]{\textcolor[rgb]{0.13,0.29,0.53}{#1}}
\newcommand{\DecValTok}[1]{\textcolor[rgb]{0.00,0.00,0.81}{#1}}
\newcommand{\DocumentationTok}[1]{\textcolor[rgb]{0.56,0.35,0.01}{\textbf{\textit{#1}}}}
\newcommand{\ErrorTok}[1]{\textcolor[rgb]{0.64,0.00,0.00}{\textbf{#1}}}
\newcommand{\ExtensionTok}[1]{#1}
\newcommand{\FloatTok}[1]{\textcolor[rgb]{0.00,0.00,0.81}{#1}}
\newcommand{\FunctionTok}[1]{\textcolor[rgb]{0.00,0.00,0.00}{#1}}
\newcommand{\ImportTok}[1]{#1}
\newcommand{\InformationTok}[1]{\textcolor[rgb]{0.56,0.35,0.01}{\textbf{\textit{#1}}}}
\newcommand{\KeywordTok}[1]{\textcolor[rgb]{0.13,0.29,0.53}{\textbf{#1}}}
\newcommand{\NormalTok}[1]{#1}
\newcommand{\OperatorTok}[1]{\textcolor[rgb]{0.81,0.36,0.00}{\textbf{#1}}}
\newcommand{\OtherTok}[1]{\textcolor[rgb]{0.56,0.35,0.01}{#1}}
\newcommand{\PreprocessorTok}[1]{\textcolor[rgb]{0.56,0.35,0.01}{\textit{#1}}}
\newcommand{\RegionMarkerTok}[1]{#1}
\newcommand{\SpecialCharTok}[1]{\textcolor[rgb]{0.00,0.00,0.00}{#1}}
\newcommand{\SpecialStringTok}[1]{\textcolor[rgb]{0.31,0.60,0.02}{#1}}
\newcommand{\StringTok}[1]{\textcolor[rgb]{0.31,0.60,0.02}{#1}}
\newcommand{\VariableTok}[1]{\textcolor[rgb]{0.00,0.00,0.00}{#1}}
\newcommand{\VerbatimStringTok}[1]{\textcolor[rgb]{0.31,0.60,0.02}{#1}}
\newcommand{\WarningTok}[1]{\textcolor[rgb]{0.56,0.35,0.01}{\textbf{\textit{#1}}}}
\usepackage{graphicx}
\makeatletter
\def\maxwidth{\ifdim\Gin@nat@width>\linewidth\linewidth\else\Gin@nat@width\fi}
\def\maxheight{\ifdim\Gin@nat@height>\textheight\textheight\else\Gin@nat@height\fi}
\makeatother
% Scale images if necessary, so that they will not overflow the page
% margins by default, and it is still possible to overwrite the defaults
% using explicit options in \includegraphics[width, height, ...]{}
\setkeys{Gin}{width=\maxwidth,height=\maxheight,keepaspectratio}
% Set default figure placement to htbp
\makeatletter
\def\fps@figure{htbp}
\makeatother
\setlength{\emergencystretch}{3em} % prevent overfull lines
\providecommand{\tightlist}{%
  \setlength{\itemsep}{0pt}\setlength{\parskip}{0pt}}
\setcounter{secnumdepth}{-\maxdimen} % remove section numbering
% Make \paragraph and \subparagraph free-standing
\ifx\paragraph\undefined\else
  \let\oldparagraph\paragraph
  \renewcommand{\paragraph}[1]{\oldparagraph{#1}\mbox{}}
\fi
\ifx\subparagraph\undefined\else
  \let\oldsubparagraph\subparagraph
  \renewcommand{\subparagraph}[1]{\oldsubparagraph{#1}\mbox{}}
\fi
\newlength{\cslhangindent}
\setlength{\cslhangindent}{1.5em}
\newlength{\csllabelwidth}
\setlength{\csllabelwidth}{3em}
\newlength{\cslentryspacingunit} % times entry-spacing
\setlength{\cslentryspacingunit}{\parskip}
\newenvironment{CSLReferences}[2] % #1 hanging-ident, #2 entry spacing
 {% don't indent paragraphs
  \setlength{\parindent}{0pt}
  % turn on hanging indent if param 1 is 1
  \ifodd #1
  \let\oldpar\par
  \def\par{\hangindent=\cslhangindent\oldpar}
  \fi
  % set entry spacing
  \setlength{\parskip}{#2\cslentryspacingunit}
 }%
 {}
\usepackage{calc}
\newcommand{\CSLBlock}[1]{#1\hfill\break}
\newcommand{\CSLLeftMargin}[1]{\parbox[t]{\csllabelwidth}{#1}}
\newcommand{\CSLRightInline}[1]{\parbox[t]{\linewidth - \csllabelwidth}{#1}\break}
\newcommand{\CSLIndent}[1]{\hspace{\cslhangindent}#1}
\ifLuaTeX
\usepackage[bidi=basic]{babel}
\else
\usepackage[bidi=default]{babel}
\fi
\babelprovide[main,import]{english}
% get rid of language-specific shorthands (see #6817):
\let\LanguageShortHands\languageshorthands
\def\languageshorthands#1{}
% Manuscript styling
\usepackage{upgreek}
\captionsetup{font=singlespacing,justification=justified}

% Table formatting
\usepackage{longtable}
\usepackage{lscape}
% \usepackage[counterclockwise]{rotating}   % Landscape page setup for large tables
\usepackage{multirow}		% Table styling
\usepackage{tabularx}		% Control Column width
\usepackage[flushleft]{threeparttable}	% Allows for three part tables with a specified notes section
\usepackage{threeparttablex}            % Lets threeparttable work with longtable

% Create new environments so endfloat can handle them
% \newenvironment{ltable}
%   {\begin{landscape}\centering\begin{threeparttable}}
%   {\end{threeparttable}\end{landscape}}
\newenvironment{lltable}{\begin{landscape}\centering\begin{ThreePartTable}}{\end{ThreePartTable}\end{landscape}}

% Enables adjusting longtable caption width to table width
% Solution found at http://golatex.de/longtable-mit-caption-so-breit-wie-die-tabelle-t15767.html
\makeatletter
\newcommand\LastLTentrywidth{1em}
\newlength\longtablewidth
\setlength{\longtablewidth}{1in}
\newcommand{\getlongtablewidth}{\begingroup \ifcsname LT@\roman{LT@tables}\endcsname \global\longtablewidth=0pt \renewcommand{\LT@entry}[2]{\global\advance\longtablewidth by ##2\relax\gdef\LastLTentrywidth{##2}}\@nameuse{LT@\roman{LT@tables}} \fi \endgroup}

% \setlength{\parindent}{0.5in}
% \setlength{\parskip}{0pt plus 0pt minus 0pt}

% Overwrite redefinition of paragraph and subparagraph by the default LaTeX template
% See https://github.com/crsh/papaja/issues/292
\makeatletter
\renewcommand{\paragraph}{\@startsection{paragraph}{4}{\parindent}%
  {0\baselineskip \@plus 0.2ex \@minus 0.2ex}%
  {-1em}%
  {\normalfont\normalsize\bfseries\itshape\typesectitle}}

\renewcommand{\subparagraph}[1]{\@startsection{subparagraph}{5}{1em}%
  {0\baselineskip \@plus 0.2ex \@minus 0.2ex}%
  {-\z@\relax}%
  {\normalfont\normalsize\itshape\hspace{\parindent}{#1}\textit{\addperi}}{\relax}}
\makeatother

% \usepackage{etoolbox}
\makeatletter
\patchcmd{\HyOrg@maketitle}
  {\section{\normalfont\normalsize\abstractname}}
  {\section*{\normalfont\normalsize\abstractname}}
  {}{\typeout{Failed to patch abstract.}}
\patchcmd{\HyOrg@maketitle}
  {\section{\protect\normalfont{\@title}}}
  {\section*{\protect\normalfont{\@title}}}
  {}{\typeout{Failed to patch title.}}
\makeatother

\usepackage{xpatch}
\makeatletter
\xapptocmd\appendix
  {\xapptocmd\section
    {\addcontentsline{toc}{section}{\appendixname\ifoneappendix\else~\theappendix\fi\\: #1}}
    {}{\InnerPatchFailed}%
  }
{}{\PatchFailed}
\keywords{keywords\newline\indent Word count: X}
\usepackage{lineno}

\linenumbers
\usepackage{csquotes}
\makeatletter
\renewcommand{\paragraph}{\@startsection{paragraph}{4}{\parindent}%
  {0\baselineskip \@plus 0.2ex \@minus 0.2ex}%
  {-1em}%
  {\normalfont\normalsize\bfseries\typesectitle}}

\renewcommand{\subparagraph}[1]{\@startsection{subparagraph}{5}{1em}%
  {0\baselineskip \@plus 0.2ex \@minus 0.2ex}%
  {-\z@\relax}%
  {\normalfont\normalsize\bfseries\itshape\hspace{\parindent}{#1}\textit{\addperi}}{\relax}}
\makeatother

\ifLuaTeX
  \usepackage{selnolig}  % disable illegal ligatures
\fi
\IfFileExists{bookmark.sty}{\usepackage{bookmark}}{\usepackage{hyperref}}
\IfFileExists{xurl.sty}{\usepackage{xurl}}{} % add URL line breaks if available
\urlstyle{same}
\hypersetup{
  pdftitle={Supplementary material: Absolute Pitch and Sound-Color-Synesthesia},
  pdfauthor={Beat Meier1, Andrew W. Ellis1, \& Solange Glasser2},
  pdflang={en-EN},
  pdfkeywords={keywords},
  hidelinks,
  pdfcreator={LaTeX via pandoc}}

\title{Supplementary material: Absolute Pitch and Sound-Color-Synesthesia}
\author{Beat Meier\textsuperscript{1}, Andrew W. Ellis\textsuperscript{1}, \& Solange Glasser\textsuperscript{2}}
\date{}


\shorttitle{Supplementary material}

\authornote{

\addORCIDlink{Andrew W. Ellis}{0000-0002-2788-936X}

The authors made the following contributions. Beat Meier: Conceptualization, Writing - Original Draft Preparation, Writing - Review \& Editing; Andrew W. Ellis: Writing - Review \& Editing, Data analysis; Solange Glasser: Conceptualization, Writing - Review \& Editing.

Correspondence concerning this article should be addressed to Beat Meier, Department of Psychology, University of Bern, Fabrikstrasse 8, 3012 Bern, Switzerland. E-mail: \href{mailto:beat.meier@unibe.ch}{\nolinkurl{beat.meier@unibe.ch}}

}

\affiliation{\vspace{0.5cm}\textsuperscript{1} Department of Psychology, University of Bern, Switzerland\\\textsuperscript{2} University of Melbourne, Australia}

\abstract{%
One or two sentences providing a \textbf{basic introduction} to the field, comprehensible to a scientist in any discipline.
}



\begin{document}
\maketitle

\hypertarget{plot-theme}{%
\section{Plot theme}\label{plot-theme}}

\begin{Shaded}
\begin{Highlighting}[]
\NormalTok{theme\_clean }\OtherTok{\textless{}{-}} \ControlFlowTok{function}\NormalTok{() \{}
  \FunctionTok{theme\_minimal}\NormalTok{(}\AttributeTok{base\_family =} \StringTok{"Helvetica"}\NormalTok{, }\AttributeTok{base\_size =} \DecValTok{12}\NormalTok{) }\SpecialCharTok{+}
    \FunctionTok{theme}\NormalTok{(}\AttributeTok{panel.grid.minor =} \FunctionTok{element\_blank}\NormalTok{(),}
          \AttributeTok{plot.title =} \FunctionTok{element\_text}\NormalTok{(}\AttributeTok{face =} \StringTok{"bold"}\NormalTok{),}
          \AttributeTok{axis.title =} \FunctionTok{element\_text}\NormalTok{(}\AttributeTok{face =} \StringTok{"bold"}\NormalTok{),}
          \AttributeTok{strip.text =} \FunctionTok{element\_text}\NormalTok{(}\AttributeTok{face =} \StringTok{"bold"}\NormalTok{, }\AttributeTok{size =} \FunctionTok{rel}\NormalTok{(}\DecValTok{1}\NormalTok{), }\AttributeTok{hjust =} \DecValTok{0}\NormalTok{),}
          \AttributeTok{strip.background =} \FunctionTok{element\_rect}\NormalTok{(}\AttributeTok{fill =} \StringTok{"grey80"}\NormalTok{, }\AttributeTok{color =} \ConstantTok{NA}\NormalTok{),}
          \CommentTok{\# strip.background = element\_rect(colour="grey80", fill="grey80"),}
          \AttributeTok{legend.title =} \FunctionTok{element\_text}\NormalTok{(}\AttributeTok{face =} \StringTok{"bold"}\NormalTok{))}
\NormalTok{\}}
\end{Highlighting}
\end{Shaded}

\hypertarget{load-data}{%
\section{Load Data}\label{load-data}}

\begin{Shaded}
\begin{Highlighting}[]
\NormalTok{d }\OtherTok{\textless{}{-}}\NormalTok{ readxl}\SpecialCharTok{::}\FunctionTok{read\_excel}\NormalTok{(}\StringTok{"./data/APfor Modelling.xlsx"}\NormalTok{) }\SpecialCharTok{|\textgreater{}}
  \FunctionTok{filter}\NormalTok{(Experimentteil }\SpecialCharTok{!=} \StringTok{"nurfarbe"}\NormalTok{) }\SpecialCharTok{|\textgreater{}}
  \FunctionTok{mutate}\NormalTok{(}
    \AttributeTok{ID =} \FunctionTok{as\_factor}\NormalTok{(Subject),}
    \AttributeTok{group3 =} \FunctionTok{as\_factor}\NormalTok{(Group),}
    \AttributeTok{syn =} \FunctionTok{as\_factor}\NormalTok{(Syn),}
    \AttributeTok{test =} \FunctionTok{as\_factor}\NormalTok{(Experimentteil),}
    \AttributeTok{time =} \FunctionTok{as.ordered}\NormalTok{(}\FunctionTok{as.numeric}\NormalTok{(test)),}
    \AttributeTok{oldnew =} \FunctionTok{as\_factor}\NormalTok{(oldnew),}
    \AttributeTok{item =} \FunctionTok{as\_factor}\NormalTok{(Item),}
    \AttributeTok{rating =} \FunctionTok{ordered}\NormalTok{(response),}
    \AttributeTok{oldnew =} \FunctionTok{fct\_recode}\NormalTok{(oldnew, }\AttributeTok{old =} \StringTok{"alt"}\NormalTok{, }\AttributeTok{new =} \StringTok{"new"}\NormalTok{),}
    \AttributeTok{oldnew =} \FunctionTok{fct\_relevel}\NormalTok{(oldnew, }\StringTok{"new"}\NormalTok{),}
    \AttributeTok{triplet =} \FunctionTok{ifelse}\NormalTok{(oldnew }\SpecialCharTok{==} \StringTok{"new"}\NormalTok{, }\SpecialCharTok{{-}}\DecValTok{1}\NormalTok{, }\DecValTok{1}\NormalTok{),}
    \AttributeTok{syn =} \FunctionTok{fct\_recode}\NormalTok{(syn, }\AttributeTok{syn =} \StringTok{"1"}\NormalTok{, }\AttributeTok{nosyn =} \StringTok{"0"}\NormalTok{)}
\NormalTok{  )}

\NormalTok{d }\OtherTok{\textless{}{-}}\NormalTok{ d }\SpecialCharTok{|\textgreater{}}
  \FunctionTok{unite}\NormalTok{(}\AttributeTok{col =}\NormalTok{ group4, group3, syn) }\SpecialCharTok{|\textgreater{}}
  \FunctionTok{transmute}\NormalTok{(}
    \AttributeTok{group4 =} \FunctionTok{as\_factor}\NormalTok{(group4),}
    \AttributeTok{group4 =} \FunctionTok{fct\_recode}\NormalTok{(group4,}
      \AttributeTok{control =} \StringTok{"KG\_nosyn"}\NormalTok{,}
      \AttributeTok{relpitch =} \StringTok{"RP\_nosyn"}\NormalTok{,}
      \AttributeTok{abspitch =} \StringTok{"AP\_nosyn"}\NormalTok{,}
      \AttributeTok{syn =} \StringTok{"AP\_syn"}
\NormalTok{    ),}
    \AttributeTok{group4 =} \FunctionTok{fct\_relevel}\NormalTok{(}
\NormalTok{      group4, }\StringTok{"control"}\NormalTok{, }\StringTok{"relpitch"}\NormalTok{,}
      \StringTok{"abspitch"}\NormalTok{, }\StringTok{"syn"}
\NormalTok{    )}
\NormalTok{  ) }\SpecialCharTok{|\textgreater{}}
  \FunctionTok{bind\_cols}\NormalTok{(d) }\SpecialCharTok{|\textgreater{}}
  \FunctionTok{mutate}\NormalTok{(}
    \AttributeTok{group3 =} \FunctionTok{fct\_recode}\NormalTok{(group3,}
      \AttributeTok{control =} \StringTok{"KG"}\NormalTok{,}
      \AttributeTok{relpitch =} \StringTok{"RP"}\NormalTok{,}
      \AttributeTok{abspitch =} \StringTok{"AP"}
\NormalTok{    ),}
    \AttributeTok{group3 =} \FunctionTok{fct\_relevel}\NormalTok{(}
\NormalTok{      group3, }\StringTok{"control"}\NormalTok{, }\StringTok{"relpitch"}\NormalTok{,}
      \StringTok{"abspitch"}
\NormalTok{    ),}
    \AttributeTok{group2 =} \FunctionTok{as\_factor}\NormalTok{(}\FunctionTok{case\_when}\NormalTok{(}
\NormalTok{      group3 }\SpecialCharTok{==} \StringTok{"abspitch"} \SpecialCharTok{|}\NormalTok{ group3 }\SpecialCharTok{==} \StringTok{"relpitch"} \SpecialCharTok{\textasciitilde{}} \StringTok{"musician"}\NormalTok{,}
\NormalTok{      group3 }\SpecialCharTok{==} \StringTok{"control"} \SpecialCharTok{\textasciitilde{}} \StringTok{"control"}
\NormalTok{    )),}
    \AttributeTok{group2 =} \FunctionTok{fct\_relevel}\NormalTok{(}
\NormalTok{      group2, }\StringTok{"control"}\NormalTok{, }\StringTok{"musician"}
\NormalTok{    ),}
    \AttributeTok{group3alt =} \FunctionTok{as\_factor}\NormalTok{(}\FunctionTok{case\_when}\NormalTok{(}
\NormalTok{      group4 }\SpecialCharTok{==} \StringTok{"abspitch"} \SpecialCharTok{|}\NormalTok{ group4 }\SpecialCharTok{==} \StringTok{"relpitch"} \SpecialCharTok{\textasciitilde{}} \StringTok{"musician"}\NormalTok{,}
\NormalTok{      group4 }\SpecialCharTok{==} \StringTok{"syn"} \SpecialCharTok{\textasciitilde{}} \StringTok{"syn"}\NormalTok{,}
\NormalTok{      group4 }\SpecialCharTok{==} \StringTok{"control"} \SpecialCharTok{\textasciitilde{}} \StringTok{"control"}
\NormalTok{    )),}
    \AttributeTok{group3alt =} \FunctionTok{fct\_relevel}\NormalTok{(}
\NormalTok{      group3alt, }\StringTok{"control"}\NormalTok{, }\StringTok{"musician"}\NormalTok{, }\StringTok{"syn"}
\NormalTok{    ),}
    \AttributeTok{test =} \FunctionTok{fct\_recode}\NormalTok{(test,}
      \AttributeTok{colors =} \StringTok{"nurfarbe2"}\NormalTok{,}
      \AttributeTok{tones =} \StringTok{"nurton"}\NormalTok{,}
      \AttributeTok{combined =} \StringTok{"beides"}
\NormalTok{    ),}
    \AttributeTok{test =} \FunctionTok{fct\_relevel}\NormalTok{(}
\NormalTok{      test, }\StringTok{"colors"}\NormalTok{, }\StringTok{"tones"}\NormalTok{, }\StringTok{"combined"}
\NormalTok{    )}
\NormalTok{  ) }\SpecialCharTok{|\textgreater{}}
  \FunctionTok{select}\NormalTok{(}
\NormalTok{    ID, group2, group3, group3alt, group4, syn, test,}
\NormalTok{    time, oldnew, triplet, item, response,}
\NormalTok{    rating}
\NormalTok{  )}



\DocumentationTok{\#\# remove subject 1305 (control) {-}\textgreater{} synaesthesia? {-}{-}{-}{-}}

\NormalTok{d }\OtherTok{\textless{}{-}}\NormalTok{ d }\SpecialCharTok{|\textgreater{}}
  \FunctionTok{filter}\NormalTok{(}\SpecialCharTok{!}\NormalTok{(ID }\SpecialCharTok{\%in\%} \DecValTok{10305}\NormalTok{)) }\SpecialCharTok{|\textgreater{}}
  \FunctionTok{mutate}\NormalTok{(}\AttributeTok{ID =} \FunctionTok{droplevels}\NormalTok{(ID))}
\end{Highlighting}
\end{Shaded}

In this study, subjects in 3 groups (non-musician controls, musicians with relative pitch, musicians with absolute pitch) gave confidence judgments on previously learned or previously unseen (old/new) triplets of stimuli. Confidence ratings were given on a 5-point response scale (1-5). Subjects were tested in three conditions (colours, tones, and colours and tones combined) with both old and new stimuli, resulting in a \(3 x 3 x 2\) mixed design.

It was subsequently discovered that 7 of the musicians with absolute pitch were also synaesthetes; this subgroup was analyzed separately, as a \(4 x 3 x 2\) design with the between factor group membership and the within factor experimental condition.

In the original three groups, there were 19, 22 and 24 subjects, respectively.

\begin{Shaded}
\begin{Highlighting}[]
\NormalTok{d }\SpecialCharTok{|\textgreater{}}
  \FunctionTok{group\_by}\NormalTok{(group3) }\SpecialCharTok{|\textgreater{}}
  \FunctionTok{summarise}\NormalTok{(}\AttributeTok{n =} \FunctionTok{n\_distinct}\NormalTok{(ID))}
\end{Highlighting}
\end{Shaded}

\begin{verbatim}
## # A tibble: 3 x 2
##   group3       n
##   <fct>    <int>
## 1 control     19
## 2 relpitch    22
## 3 abspitch    24
\end{verbatim}

\begin{Shaded}
\begin{Highlighting}[]
\CommentTok{\# d |\textgreater{}}
\CommentTok{\#   group\_by(group3) \%\textgreater{}\%}
\CommentTok{\#   distinct(ID) \%\textgreater{}\%}
\CommentTok{\#   count()}
\end{Highlighting}
\end{Shaded}

Out of the \(24\) subjects with absolute pitch, \(7\) were synaesthetes.

\begin{Shaded}
\begin{Highlighting}[]
\NormalTok{d }\SpecialCharTok{|\textgreater{}}
  \FunctionTok{group\_by}\NormalTok{(group4) }\SpecialCharTok{|\textgreater{}}
  \FunctionTok{summarise}\NormalTok{(}\AttributeTok{n =} \FunctionTok{n\_distinct}\NormalTok{(ID))}
\end{Highlighting}
\end{Shaded}

\begin{verbatim}
## # A tibble: 4 x 2
##   group4       n
##   <fct>    <int>
## 1 control     19
## 2 relpitch    22
## 3 abspitch    17
## 4 syn          7
\end{verbatim}

\hypertarget{learning-score}{%
\section{Learning Score}\label{learning-score}}

In this type of experiment, the traditional approach is to treat the ordinal response as a continuous variable, and to compute the mean response category for old and new items for each combination of test type/group, and then compute a learning score as the difference in mean response to old and new items.

However, treating an ordinal response as a continuous variable is associated with several problems (Liddell \& Kruschke, 2018). While the categories have an ordering, but it is unknown what the \emph{psychological distance} between those categories is, and whether distances between categories are the same across subjects. An alternative approach is to use an ordered regression model, in which it is assumed that the observed variable \(Y\) originates from the categorization of a latent continuous variable \(\tilde{Y}\). There are \(K\) thresholds \(\tau_k\) , which partition \(\tilde{Y}\) into \(K+1\) observable, ordered categories of \(Y\).

\hypertarget{main-points}{%
\section{Main points}\label{main-points}}

\begin{enumerate}
\def\labelenumi{\arabic{enumi})}
\item
  Musicians (with both relative pitch and absolute pitch) are better at
  learning triplets than non-musicians (controls).
\item
  There is no difference between absolute pitch and relative pitch. Having
  absolute pitch confers no advantage over relative pitch.
\item
  Any advantage in the recognition task is due to synaesthesia.
\item
  Absolute pitch confers an advantage in associative learning (color-tone),
  compared to relative pitch.
\item
  Dissociation: There is no difference between abs pitch and synaesthesia in the colour memory task.
\item
  Memory of colour is very accurate (but it is not synaesthesia that
  leads to very good color reproducibility).
\end{enumerate}

\hypertarget{exploratory-data-analysis}{%
\section{Exploratory Data Analysis}\label{exploratory-data-analysis}}

\begin{Shaded}
\begin{Highlighting}[]
\NormalTok{se }\OtherTok{\textless{}{-}} \ControlFlowTok{function}\NormalTok{(x) }\FunctionTok{sd}\NormalTok{(x) }\SpecialCharTok{/} \FunctionTok{sqrt}\NormalTok{(}\FunctionTok{length}\NormalTok{(x))}
\NormalTok{funs }\OtherTok{\textless{}{-}} \FunctionTok{list}\NormalTok{(}\AttributeTok{mean =}\NormalTok{ mean, }\AttributeTok{sd =}\NormalTok{ sd, }\AttributeTok{se =}\NormalTok{ se)}
\end{Highlighting}
\end{Shaded}

\begin{Shaded}
\begin{Highlighting}[]
\NormalTok{point\_estimates }\OtherTok{\textless{}{-}}\NormalTok{ d }\SpecialCharTok{|\textgreater{}}
  \FunctionTok{group\_by}\NormalTok{(ID, group4, test, oldnew) }\SpecialCharTok{|\textgreater{}}
  \FunctionTok{summarise}\NormalTok{(}\AttributeTok{mean =} \FunctionTok{mean}\NormalTok{(response)) }\SpecialCharTok{|\textgreater{}}
  \FunctionTok{spread}\NormalTok{(oldnew, mean) }\SpecialCharTok{|\textgreater{}}
  \FunctionTok{mutate}\NormalTok{(}\AttributeTok{score =}\NormalTok{ old }\SpecialCharTok{{-}}\NormalTok{ new)}


\NormalTok{point\_estimates }\SpecialCharTok{|\textgreater{}}
  \FunctionTok{ggplot}\NormalTok{(}\FunctionTok{aes}\NormalTok{(}\AttributeTok{x =}\NormalTok{ test, }\AttributeTok{y =}\NormalTok{ score, }\AttributeTok{color =}\NormalTok{ group4)) }\SpecialCharTok{+}
  \FunctionTok{geom\_line}\NormalTok{(}\FunctionTok{aes}\NormalTok{(}\AttributeTok{group =}\NormalTok{ ID)) }\SpecialCharTok{+}
  \FunctionTok{geom\_point}\NormalTok{() }\SpecialCharTok{+}
  \FunctionTok{facet\_wrap}\NormalTok{(}\SpecialCharTok{\textasciitilde{}}\NormalTok{group4) }\SpecialCharTok{+}
  \CommentTok{\# scale\_color\_okabe\_ito() +}
  \FunctionTok{scale\_color\_viridis\_d}\NormalTok{(}\AttributeTok{direction =} \DecValTok{1}\NormalTok{, }\AttributeTok{option =} \StringTok{"C"}\NormalTok{, }\AttributeTok{end =}\NormalTok{ .}\DecValTok{80}\NormalTok{) }\SpecialCharTok{+}
  \FunctionTok{labs}\NormalTok{(}\AttributeTok{x =} \StringTok{"Test"}\NormalTok{, }\AttributeTok{y =} \StringTok{"Learning score"}\NormalTok{, }\AttributeTok{color =} \StringTok{"Group"}\NormalTok{) }\SpecialCharTok{+}
  \FunctionTok{theme\_clean}\NormalTok{()}
\end{Highlighting}
\end{Shaded}

\includegraphics{supplementary_files/figure-latex/unnamed-chunk-5-1.pdf}

\begin{Shaded}
\begin{Highlighting}[]
\NormalTok{d }\SpecialCharTok{\%\textgreater{}\%}
  \FunctionTok{group\_by}\NormalTok{(ID, group4, test, oldnew) }\SpecialCharTok{\%\textgreater{}\%}
  \FunctionTok{summarise}\NormalTok{(}\AttributeTok{mean =} \FunctionTok{mean}\NormalTok{(response)) }\SpecialCharTok{\%\textgreater{}\%}
  \FunctionTok{ggplot}\NormalTok{(}\FunctionTok{aes}\NormalTok{(}\AttributeTok{x =}\NormalTok{ test, }\AttributeTok{y =}\NormalTok{ mean, }\AttributeTok{color =}\NormalTok{ oldnew)) }\SpecialCharTok{+}
  \FunctionTok{geom\_line}\NormalTok{(}\FunctionTok{aes}\NormalTok{(}\AttributeTok{group =} \FunctionTok{interaction}\NormalTok{(ID, oldnew))) }\SpecialCharTok{+}
  \FunctionTok{geom\_point}\NormalTok{() }\SpecialCharTok{+}
  \FunctionTok{facet\_wrap}\NormalTok{(}\SpecialCharTok{\textasciitilde{}}\NormalTok{group4) }\SpecialCharTok{+}
  \FunctionTok{scale\_color\_viridis\_d}\NormalTok{(}\AttributeTok{direction =} \DecValTok{1}\NormalTok{, }\AttributeTok{option =} \StringTok{"C"}\NormalTok{, }\AttributeTok{end =}\NormalTok{ .}\DecValTok{80}\NormalTok{) }\SpecialCharTok{+}
  \FunctionTok{labs}\NormalTok{(}\AttributeTok{x =} \StringTok{"Test"}\NormalTok{, }\AttributeTok{y =} \StringTok{"Mean rating"}\NormalTok{, }\AttributeTok{color =} \StringTok{"Triplet"}\NormalTok{) }\SpecialCharTok{+}
  \FunctionTok{theme\_clean}\NormalTok{()}
\end{Highlighting}
\end{Shaded}

\includegraphics{supplementary_files/figure-latex/unnamed-chunk-6-1.pdf}

In the following, we compute the learning scores for each individual subject in each test condition as the the difference in mean response to old and new items, \texttt{mean(old)\ -\ mean(new)}. Positive learning scores thus indicate that the subject gave higher ratings to previously seen triplets than to unseen triplets. Higher scores are interpreted as greater being due to greater learning of the triplets; subjects are able to confidently state that they have previously seen old triplets, whilst being able to reject unseen triplets.

All figures show mean learning scores in all three conditions, aggregated over subjects, with within-subjects confidence intervals (Morey, 2008).

\hypertarget{musicians-vs-controls}{%
\subsection{Musicians vs controls}\label{musicians-vs-controls}}

\begin{Shaded}
\begin{Highlighting}[]
\NormalTok{sdtdata\_2 }\OtherTok{\textless{}{-}}\NormalTok{ d }\SpecialCharTok{|\textgreater{}}
  \FunctionTok{group\_by}\NormalTok{(ID, group2, test, oldnew) }\SpecialCharTok{|\textgreater{}}
  \FunctionTok{summarise}\NormalTok{(}\AttributeTok{mean =} \FunctionTok{mean}\NormalTok{(response)) }\SpecialCharTok{|\textgreater{}}
  \FunctionTok{pivot\_wider}\NormalTok{(}\AttributeTok{names\_from =}\NormalTok{ oldnew, }\AttributeTok{values\_from =}\NormalTok{ mean) }\SpecialCharTok{|\textgreater{}}
  \FunctionTok{mutate}\NormalTok{(}\AttributeTok{score =}\NormalTok{ old }\SpecialCharTok{{-}}\NormalTok{ new)}

\NormalTok{sdtdata\_2\_agg }\OtherTok{\textless{}{-}}\NormalTok{ sdtdata\_2 }\SpecialCharTok{|\textgreater{}}
  \FunctionTok{drop\_na}\NormalTok{() }\SpecialCharTok{|\textgreater{}}
  \FunctionTok{group\_by}\NormalTok{(group2, test) }\SpecialCharTok{|\textgreater{}}
  \FunctionTok{summarise}\NormalTok{(}\FunctionTok{across}\NormalTok{(score, funs, }\AttributeTok{.names =} \StringTok{"\{.fn\}"}\NormalTok{))}

\NormalTok{sdtdata\_2\_agg\_within }\OtherTok{\textless{}{-}}\NormalTok{ sdtdata\_2 }\SpecialCharTok{|\textgreater{}}
\NormalTok{  Rmisc}\SpecialCharTok{::}\FunctionTok{summarySEwithin}\NormalTok{(}
    \AttributeTok{measurevar =} \StringTok{"score"}\NormalTok{,}
    \AttributeTok{betweenvars =} \StringTok{"group2"}\NormalTok{,}
    \AttributeTok{withinvars =} \StringTok{"test"}\NormalTok{,}
    \AttributeTok{idvar =} \StringTok{"ID"}\NormalTok{,}
    \AttributeTok{na.rm =} \ConstantTok{FALSE}\NormalTok{,}
    \AttributeTok{conf.interval =} \FloatTok{0.95}
\NormalTok{  )}

\NormalTok{sdtdata\_2\_agg\_within }\OtherTok{\textless{}{-}}\NormalTok{ sdtdata\_2\_agg\_within }\SpecialCharTok{|\textgreater{}}
  \FunctionTok{mutate}\NormalTok{(}\AttributeTok{mean =} \FunctionTok{pull}\NormalTok{(sdtdata\_2\_agg, mean))}

\NormalTok{sdtdata\_2\_agg\_within }\SpecialCharTok{|\textgreater{}}
  \FunctionTok{ggplot}\NormalTok{(}\FunctionTok{aes}\NormalTok{(}\AttributeTok{x =}\NormalTok{ test, }\AttributeTok{y =}\NormalTok{ mean, }\AttributeTok{color =}\NormalTok{ group2)) }\SpecialCharTok{+}
  \FunctionTok{geom\_line}\NormalTok{(}\FunctionTok{aes}\NormalTok{(}\AttributeTok{group =}\NormalTok{ group2), }\AttributeTok{linewidth =} \FloatTok{1.5}\NormalTok{, }\AttributeTok{linetype =} \StringTok{"dashed"}\NormalTok{) }\SpecialCharTok{+}
  \FunctionTok{geom\_point}\NormalTok{(}\AttributeTok{size =} \DecValTok{4}\NormalTok{) }\SpecialCharTok{+}
  \FunctionTok{geom\_errorbar}\NormalTok{(}
    \FunctionTok{aes}\NormalTok{(}
      \AttributeTok{ymin =}\NormalTok{ mean }\SpecialCharTok{{-}}\NormalTok{ ci,}
      \AttributeTok{ymax =}\NormalTok{ mean }\SpecialCharTok{+}\NormalTok{ ci}
\NormalTok{    ),}
    \AttributeTok{width =} \FloatTok{0.1}\NormalTok{, }\AttributeTok{linewidth =} \FloatTok{1.5}
\NormalTok{  ) }\SpecialCharTok{+}
  \FunctionTok{scale\_color\_viridis\_d}\NormalTok{(}\AttributeTok{direction =} \SpecialCharTok{{-}}\DecValTok{1}\NormalTok{, }\AttributeTok{option =} \StringTok{"C"}\NormalTok{, }\AttributeTok{end =}\NormalTok{ .}\DecValTok{85}\NormalTok{) }\SpecialCharTok{+}
  \FunctionTok{ylim}\NormalTok{(}\DecValTok{0}\NormalTok{, }\DecValTok{3}\NormalTok{) }\SpecialCharTok{+}
  \FunctionTok{labs}\NormalTok{(}\AttributeTok{x =} \StringTok{"Test"}\NormalTok{, }\AttributeTok{y =} \StringTok{"Learning score"}\NormalTok{, }\AttributeTok{color =} \StringTok{"Group"}\NormalTok{) }\SpecialCharTok{+}
  \FunctionTok{theme\_clean}\NormalTok{() }\SpecialCharTok{+}
  \FunctionTok{guides}\NormalTok{(}\AttributeTok{color =} \FunctionTok{guide\_legend}\NormalTok{(}
    \AttributeTok{title =} \StringTok{"Group"}\NormalTok{,}
    \AttributeTok{title.position =} \StringTok{"top"}
\NormalTok{  ))}
\end{Highlighting}
\end{Shaded}

\includegraphics{supplementary_files/figure-latex/unnamed-chunk-7-1.pdf}

Musicians, consisting of the groups with relative and absolute pitch (also containing the synaesthets) are clearly better at all three recognition task than controls. What is also noticeable is that musicians perform better in the tasks involving tones, whereas controls need both tones and colours combined to perform better.

\hypertarget{absolute-pitch-relative-pitch-and-controls}{%
\subsection{Absolute pitch, relative pitch and controls}\label{absolute-pitch-relative-pitch-and-controls}}

\begin{Shaded}
\begin{Highlighting}[]
\NormalTok{sdtdata\_3 }\OtherTok{\textless{}{-}}\NormalTok{ d }\SpecialCharTok{|\textgreater{}}
  \FunctionTok{group\_by}\NormalTok{(ID, group3, test, oldnew) }\SpecialCharTok{|\textgreater{}}
  \FunctionTok{summarise}\NormalTok{(}\AttributeTok{mean =} \FunctionTok{mean}\NormalTok{(response)) }\SpecialCharTok{|\textgreater{}}
  \FunctionTok{pivot\_wider}\NormalTok{(}\AttributeTok{names\_from =}\NormalTok{ oldnew, }\AttributeTok{values\_from =}\NormalTok{ mean) }\SpecialCharTok{|\textgreater{}}
  \FunctionTok{mutate}\NormalTok{(}\AttributeTok{score =}\NormalTok{ old }\SpecialCharTok{{-}}\NormalTok{ new)}

\NormalTok{sdtdata\_3\_agg }\OtherTok{\textless{}{-}}\NormalTok{ sdtdata\_3 }\SpecialCharTok{|\textgreater{}}
  \FunctionTok{drop\_na}\NormalTok{() }\SpecialCharTok{|\textgreater{}}
  \FunctionTok{group\_by}\NormalTok{(group3, test) }\SpecialCharTok{|\textgreater{}}
  \FunctionTok{summarise}\NormalTok{(}\FunctionTok{across}\NormalTok{(score, funs,}
    \AttributeTok{.names =} \StringTok{"\{.fn\}"}
\NormalTok{  ))}

\NormalTok{sdtdata\_3\_agg\_within }\OtherTok{\textless{}{-}}\NormalTok{ sdtdata\_3 }\SpecialCharTok{|\textgreater{}}
\NormalTok{  Rmisc}\SpecialCharTok{::}\FunctionTok{summarySEwithin}\NormalTok{(}
    \AttributeTok{measurevar =} \StringTok{"score"}\NormalTok{,}
    \AttributeTok{betweenvars =} \StringTok{"group3"}\NormalTok{,}
    \AttributeTok{withinvars =} \StringTok{"test"}\NormalTok{,}
    \AttributeTok{idvar =} \StringTok{"ID"}\NormalTok{,}
    \AttributeTok{na.rm =} \ConstantTok{FALSE}\NormalTok{,}
    \AttributeTok{conf.interval =} \FloatTok{0.95}
\NormalTok{  )}

\NormalTok{sdtdata\_3\_agg\_within }\OtherTok{\textless{}{-}}\NormalTok{ sdtdata\_3\_agg\_within }\SpecialCharTok{|\textgreater{}}
  \FunctionTok{mutate}\NormalTok{(}\AttributeTok{mean =} \FunctionTok{pull}\NormalTok{(sdtdata\_3\_agg, mean))}

\NormalTok{sdtdata\_3\_agg\_within }\SpecialCharTok{|\textgreater{}}
  \FunctionTok{ggplot}\NormalTok{(}\FunctionTok{aes}\NormalTok{(}\AttributeTok{x =}\NormalTok{ test, }\AttributeTok{y =}\NormalTok{ mean, }\AttributeTok{color =}\NormalTok{ group3)) }\SpecialCharTok{+}
  \FunctionTok{geom\_line}\NormalTok{(}\FunctionTok{aes}\NormalTok{(}\AttributeTok{group =}\NormalTok{ group3), }\AttributeTok{linewidth =} \FloatTok{1.5}\NormalTok{, }\AttributeTok{linetype =} \StringTok{"dashed"}\NormalTok{) }\SpecialCharTok{+}
  \FunctionTok{geom\_point}\NormalTok{(}\AttributeTok{size =} \DecValTok{4}\NormalTok{) }\SpecialCharTok{+}
  \FunctionTok{geom\_errorbar}\NormalTok{(}
    \FunctionTok{aes}\NormalTok{(}
      \AttributeTok{ymin =}\NormalTok{ mean }\SpecialCharTok{{-}}\NormalTok{ ci,}
      \AttributeTok{ymax =}\NormalTok{ mean }\SpecialCharTok{+}\NormalTok{ ci}
\NormalTok{    ),}
    \AttributeTok{width =} \FloatTok{0.1}\NormalTok{, }\AttributeTok{linewidth =} \FloatTok{1.5}
\NormalTok{  ) }\SpecialCharTok{+}
  \FunctionTok{scale\_color\_viridis\_d}\NormalTok{(}\AttributeTok{direction =} \SpecialCharTok{{-}}\DecValTok{1}\NormalTok{, }\AttributeTok{option =} \StringTok{"C"}\NormalTok{, }\AttributeTok{end =}\NormalTok{ .}\DecValTok{85}\NormalTok{) }\SpecialCharTok{+}
  \FunctionTok{ylim}\NormalTok{(}\DecValTok{0}\NormalTok{, }\DecValTok{3}\NormalTok{) }\SpecialCharTok{+}
  \FunctionTok{labs}\NormalTok{(}\AttributeTok{x =} \StringTok{"Test"}\NormalTok{, }\AttributeTok{y =} \StringTok{"Learning score"}\NormalTok{, }\AttributeTok{color =} \StringTok{"Group"}\NormalTok{) }\SpecialCharTok{+}
  \FunctionTok{theme\_clean}\NormalTok{() }\SpecialCharTok{+}
  \FunctionTok{guides}\NormalTok{(}\AttributeTok{color =} \FunctionTok{guide\_legend}\NormalTok{(}
    \AttributeTok{title =} \StringTok{"Group"}\NormalTok{,}
    \AttributeTok{title.position =} \StringTok{"top"}
\NormalTok{  ))}
\end{Highlighting}
\end{Shaded}

\includegraphics{supplementary_files/figure-latex/unnamed-chunk-8-1.pdf}

\hypertarget{musicians-synaesthetes-controls}{%
\subsection{Musicians, synaesthetes, controls}\label{musicians-synaesthetes-controls}}

\begin{Shaded}
\begin{Highlighting}[]
\NormalTok{sdtdata\_3alt }\OtherTok{\textless{}{-}}\NormalTok{ d }\SpecialCharTok{|\textgreater{}}
  \FunctionTok{group\_by}\NormalTok{(ID, group3alt, test, oldnew) }\SpecialCharTok{|\textgreater{}}
  \FunctionTok{summarise}\NormalTok{(}\AttributeTok{mean =} \FunctionTok{mean}\NormalTok{(response)) }\SpecialCharTok{|\textgreater{}}
  \FunctionTok{pivot\_wider}\NormalTok{(}\AttributeTok{names\_from =}\NormalTok{ oldnew, }\AttributeTok{values\_from =}\NormalTok{ mean) }\SpecialCharTok{|\textgreater{}}
  \FunctionTok{mutate}\NormalTok{(}\AttributeTok{score =}\NormalTok{ old }\SpecialCharTok{{-}}\NormalTok{ new)}

\NormalTok{sdtdata\_3alt\_agg }\OtherTok{\textless{}{-}}\NormalTok{ sdtdata\_3alt }\SpecialCharTok{|\textgreater{}}
  \FunctionTok{drop\_na}\NormalTok{() }\SpecialCharTok{|\textgreater{}}
  \FunctionTok{group\_by}\NormalTok{(group3alt, test) }\SpecialCharTok{|\textgreater{}}
  \FunctionTok{summarise}\NormalTok{(}\FunctionTok{across}\NormalTok{(score, funs,}
    \AttributeTok{.names =} \StringTok{"\{.fn\}"}
\NormalTok{  ))}

\NormalTok{sdtdata\_3alt\_agg\_within }\OtherTok{\textless{}{-}}\NormalTok{ sdtdata\_3alt }\SpecialCharTok{|\textgreater{}}
\NormalTok{  Rmisc}\SpecialCharTok{::}\FunctionTok{summarySEwithin}\NormalTok{(}
    \AttributeTok{measurevar =} \StringTok{"score"}\NormalTok{,}
    \AttributeTok{betweenvars =} \StringTok{"group3alt"}\NormalTok{,}
    \AttributeTok{withinvars =} \StringTok{"test"}\NormalTok{,}
    \AttributeTok{idvar =} \StringTok{"ID"}\NormalTok{,}
    \AttributeTok{na.rm =} \ConstantTok{FALSE}\NormalTok{,}
    \AttributeTok{conf.interval =} \FloatTok{0.95}
\NormalTok{  )}

\NormalTok{sdtdata\_3alt\_agg\_within }\OtherTok{\textless{}{-}}\NormalTok{ sdtdata\_3alt\_agg\_within }\SpecialCharTok{|\textgreater{}}
  \FunctionTok{mutate}\NormalTok{(}\AttributeTok{mean =} \FunctionTok{pull}\NormalTok{(sdtdata\_3alt\_agg, mean))}

\NormalTok{sdtdata\_3alt\_agg\_within }\SpecialCharTok{|\textgreater{}}
  \FunctionTok{ggplot}\NormalTok{(}\FunctionTok{aes}\NormalTok{(}\AttributeTok{x =}\NormalTok{ test, }\AttributeTok{y =}\NormalTok{ mean, }\AttributeTok{color =}\NormalTok{ group3alt)) }\SpecialCharTok{+}
  \FunctionTok{geom\_line}\NormalTok{(}\FunctionTok{aes}\NormalTok{(}\AttributeTok{group =}\NormalTok{ group3alt), }\AttributeTok{linewidth =} \FloatTok{1.5}\NormalTok{, }\AttributeTok{linetype =} \StringTok{"dashed"}\NormalTok{) }\SpecialCharTok{+}
  \FunctionTok{geom\_point}\NormalTok{(}\AttributeTok{size =} \DecValTok{4}\NormalTok{) }\SpecialCharTok{+}
  \FunctionTok{geom\_errorbar}\NormalTok{(}
    \FunctionTok{aes}\NormalTok{(}
      \AttributeTok{ymin =}\NormalTok{ mean }\SpecialCharTok{{-}}\NormalTok{ ci,}
      \AttributeTok{ymax =}\NormalTok{ mean }\SpecialCharTok{+}\NormalTok{ ci}
\NormalTok{    ),}
    \AttributeTok{width =} \FloatTok{0.1}\NormalTok{, }\AttributeTok{linewidth =} \FloatTok{1.5}
\NormalTok{  ) }\SpecialCharTok{+}
  \FunctionTok{scale\_color\_viridis\_d}\NormalTok{(}\AttributeTok{direction =} \SpecialCharTok{{-}}\DecValTok{1}\NormalTok{, }\AttributeTok{option =} \StringTok{"C"}\NormalTok{, }\AttributeTok{end =}\NormalTok{ .}\DecValTok{85}\NormalTok{) }\SpecialCharTok{+}
  \FunctionTok{ylim}\NormalTok{(}\DecValTok{0}\NormalTok{, }\DecValTok{3}\NormalTok{) }\SpecialCharTok{+}
  \FunctionTok{labs}\NormalTok{(}\AttributeTok{x =} \StringTok{"Test"}\NormalTok{, }\AttributeTok{y =} \StringTok{"Learning score"}\NormalTok{, }\AttributeTok{color =} \StringTok{"Group"}\NormalTok{) }\SpecialCharTok{+}
  \FunctionTok{theme\_clean}\NormalTok{() }\SpecialCharTok{+}
  \FunctionTok{guides}\NormalTok{(}\AttributeTok{color =} \FunctionTok{guide\_legend}\NormalTok{(}
    \AttributeTok{title =} \StringTok{"Group"}\NormalTok{,}
    \AttributeTok{title.position =} \StringTok{"top"}
\NormalTok{  ))}
\end{Highlighting}
\end{Shaded}

\includegraphics{supplementary_files/figure-latex/unnamed-chunk-9-1.pdf}

\hypertarget{synaesthetes-absolute-pitch-relative-pitch-and-controls}{%
\subsection{Synaesthetes, absolute pitch, relative pitch and controls}\label{synaesthetes-absolute-pitch-relative-pitch-and-controls}}

\begin{Shaded}
\begin{Highlighting}[]
\NormalTok{sdtdata\_4 }\OtherTok{\textless{}{-}}\NormalTok{ d }\SpecialCharTok{|\textgreater{}}
  \FunctionTok{group\_by}\NormalTok{(ID, group4, test, oldnew) }\SpecialCharTok{|\textgreater{}}
  \FunctionTok{summarise}\NormalTok{(}\AttributeTok{mean =} \FunctionTok{mean}\NormalTok{(response)) }\SpecialCharTok{|\textgreater{}}
  \FunctionTok{pivot\_wider}\NormalTok{(}\AttributeTok{names\_from =}\NormalTok{ oldnew, }\AttributeTok{values\_from =}\NormalTok{ mean) }\SpecialCharTok{|\textgreater{}}
  \FunctionTok{mutate}\NormalTok{(}\AttributeTok{score =}\NormalTok{ old }\SpecialCharTok{{-}}\NormalTok{ new)}

\NormalTok{sdtdata\_4\_agg }\OtherTok{\textless{}{-}}\NormalTok{ sdtdata\_4 }\SpecialCharTok{|\textgreater{}}
  \FunctionTok{drop\_na}\NormalTok{() }\SpecialCharTok{|\textgreater{}}
  \FunctionTok{group\_by}\NormalTok{(group4, test) }\SpecialCharTok{|\textgreater{}}
  \FunctionTok{summarise}\NormalTok{(}\FunctionTok{across}\NormalTok{(score, funs,}
    \AttributeTok{.names =} \StringTok{"\{.fn\}"}
\NormalTok{  ))}

\NormalTok{sdtdata\_4\_agg\_within }\OtherTok{\textless{}{-}}\NormalTok{ sdtdata\_4 }\SpecialCharTok{|\textgreater{}}
\NormalTok{  Rmisc}\SpecialCharTok{::}\FunctionTok{summarySEwithin}\NormalTok{(}
    \AttributeTok{measurevar =} \StringTok{"score"}\NormalTok{,}
    \AttributeTok{betweenvars =} \StringTok{"group4"}\NormalTok{,}
    \AttributeTok{withinvars =} \StringTok{"test"}\NormalTok{,}
    \AttributeTok{idvar =} \StringTok{"ID"}\NormalTok{,}
    \AttributeTok{na.rm =} \ConstantTok{FALSE}\NormalTok{,}
    \AttributeTok{conf.interval =} \FloatTok{0.95}
\NormalTok{  )}

\NormalTok{sdtdata\_4\_agg\_within }\OtherTok{\textless{}{-}}\NormalTok{ sdtdata\_4\_agg\_within }\SpecialCharTok{|\textgreater{}}
  \FunctionTok{mutate}\NormalTok{(}\AttributeTok{mean =} \FunctionTok{pull}\NormalTok{(sdtdata\_4\_agg, mean))}

\NormalTok{sdtdata\_4\_agg\_within }\SpecialCharTok{|\textgreater{}}
  \FunctionTok{ggplot}\NormalTok{(}\FunctionTok{aes}\NormalTok{(}\AttributeTok{x =}\NormalTok{ test, }\AttributeTok{y =}\NormalTok{ mean, }\AttributeTok{color =}\NormalTok{ group4)) }\SpecialCharTok{+}
  \FunctionTok{geom\_line}\NormalTok{(}\FunctionTok{aes}\NormalTok{(}\AttributeTok{group =}\NormalTok{ group4), }\AttributeTok{linewidth =} \FloatTok{1.5}\NormalTok{, }\AttributeTok{linetype =} \StringTok{"dashed"}\NormalTok{) }\SpecialCharTok{+}
  \FunctionTok{geom\_point}\NormalTok{(}\AttributeTok{size =} \DecValTok{4}\NormalTok{) }\SpecialCharTok{+}
  \FunctionTok{geom\_errorbar}\NormalTok{(}
    \FunctionTok{aes}\NormalTok{(}
      \AttributeTok{ymin =}\NormalTok{ mean }\SpecialCharTok{{-}}\NormalTok{ ci,}
      \AttributeTok{ymax =}\NormalTok{ mean }\SpecialCharTok{+}\NormalTok{ ci}
\NormalTok{    ),}
    \AttributeTok{width =} \FloatTok{0.1}\NormalTok{, }\AttributeTok{linewidth =} \FloatTok{1.5}
\NormalTok{  ) }\SpecialCharTok{+}
  \FunctionTok{scale\_color\_viridis\_d}\NormalTok{(}\AttributeTok{direction =} \SpecialCharTok{{-}}\DecValTok{1}\NormalTok{, }\AttributeTok{option =} \StringTok{"C"}\NormalTok{, }\AttributeTok{end =}\NormalTok{ .}\DecValTok{85}\NormalTok{) }\SpecialCharTok{+}
  \FunctionTok{ylim}\NormalTok{(}\DecValTok{0}\NormalTok{, }\DecValTok{3}\NormalTok{) }\SpecialCharTok{+}
  \FunctionTok{labs}\NormalTok{(}\AttributeTok{x =} \StringTok{""}\NormalTok{, }\AttributeTok{y =} \StringTok{"Learning score"}\NormalTok{, }\AttributeTok{color =} \StringTok{"Group"}\NormalTok{) }\SpecialCharTok{+}
  \FunctionTok{theme\_clean}\NormalTok{() }\SpecialCharTok{+}
  \FunctionTok{guides}\NormalTok{(}\AttributeTok{color =} \FunctionTok{guide\_legend}\NormalTok{(}
    \AttributeTok{title =} \StringTok{"Group"}\NormalTok{,}
    \AttributeTok{title.position =} \StringTok{"top"}
\NormalTok{  ))}
\end{Highlighting}
\end{Shaded}

\includegraphics{supplementary_files/figure-latex/unnamed-chunk-10-1.pdf}

\hypertarget{regression-models}{%
\section{Regression models}\label{regression-models}}

FIrst, we create a data set by taking the mean rating for each person in each test for both old and new items. Then we compute the \emph{learning score} by subtracting the mean rating for new items from the mean rating for old items.

\begin{Shaded}
\begin{Highlighting}[]
\NormalTok{dd }\OtherTok{\textless{}{-}}\NormalTok{ d }\SpecialCharTok{|\textgreater{}}
  \FunctionTok{group\_by}\NormalTok{(ID, group4, test, oldnew) }\SpecialCharTok{|\textgreater{}}
  \FunctionTok{summarise}\NormalTok{(}\AttributeTok{mean =} \FunctionTok{mean}\NormalTok{(response)) }\SpecialCharTok{|\textgreater{}}
  \FunctionTok{ungroup}\NormalTok{() }\SpecialCharTok{|\textgreater{}} 
  \FunctionTok{pivot\_wider}\NormalTok{(}\AttributeTok{names\_from =}\NormalTok{ oldnew, }\AttributeTok{values\_from =}\NormalTok{ mean) }\SpecialCharTok{|\textgreater{}}
  \FunctionTok{mutate}\NormalTok{(}\AttributeTok{score =}\NormalTok{ old }\SpecialCharTok{{-}}\NormalTok{ new) }
\end{Highlighting}
\end{Shaded}

The learning score is bounded in \([-4, 4]\). We will initially use a linear regression to model the outcome, a beta regression would be a better choice for a bounded variable.

\begin{Shaded}
\begin{Highlighting}[]
\NormalTok{normalize }\OtherTok{\textless{}{-}} \ControlFlowTok{function}\NormalTok{(x, }\AttributeTok{max\_x =} \DecValTok{4}\NormalTok{) \{}
\NormalTok{  min\_x }\OtherTok{\textless{}{-}} \SpecialCharTok{{-}}\NormalTok{max\_x}
\NormalTok{  (x }\SpecialCharTok{{-}}\NormalTok{ min\_x)}\SpecialCharTok{/}\NormalTok{(max\_x }\SpecialCharTok{{-}}\NormalTok{ min\_x)}
\NormalTok{  \}}

\NormalTok{dd }\OtherTok{\textless{}{-}}\NormalTok{ dd }\SpecialCharTok{|\textgreater{}} 
  \FunctionTok{mutate}\NormalTok{(}\AttributeTok{normscore =} \FunctionTok{normalize}\NormalTok{(score))}
\end{Highlighting}
\end{Shaded}

\hypertarget{groups}{%
\subsection{2 groups}\label{groups}}

\begin{Shaded}
\begin{Highlighting}[]
\NormalTok{priors }\OtherTok{\textless{}{-}} \FunctionTok{prior}\NormalTok{(}\FunctionTok{normal}\NormalTok{(}\DecValTok{0}\NormalTok{, }\DecValTok{1}\NormalTok{), }\AttributeTok{class =}\NormalTok{ Intercept) }\SpecialCharTok{+}
  \FunctionTok{prior}\NormalTok{(}\FunctionTok{normal}\NormalTok{(}\DecValTok{0}\NormalTok{, }\DecValTok{1}\NormalTok{), }\AttributeTok{class =}\NormalTok{ b) }\SpecialCharTok{+}
  \FunctionTok{prior}\NormalTok{(}\FunctionTok{student\_t}\NormalTok{(}\DecValTok{3}\NormalTok{, }\DecValTok{0}\NormalTok{, }\DecValTok{1}\NormalTok{), }\AttributeTok{class =}\NormalTok{ sd, }\AttributeTok{group =}\NormalTok{ ID) }

\NormalTok{fit\_ls\_2\_groups }\OtherTok{\textless{}{-}} \FunctionTok{brm}\NormalTok{(score }\SpecialCharTok{\textasciitilde{}}\NormalTok{ group2 }\SpecialCharTok{*}\NormalTok{ test }\SpecialCharTok{+}\NormalTok{ (}\DecValTok{1} \SpecialCharTok{|}\NormalTok{ ID),}
                      \AttributeTok{family =}\NormalTok{ gaussian,}
                      \AttributeTok{prior =}\NormalTok{ priors,}
                      \AttributeTok{data =}\NormalTok{ sdtdata\_2,}
                      \AttributeTok{chains =} \DecValTok{4}\NormalTok{, }\AttributeTok{iter =} \DecValTok{2000}\NormalTok{, }\AttributeTok{cores =} \DecValTok{4}\NormalTok{,}
                      \AttributeTok{backend =} \StringTok{"cmdstanr"}\NormalTok{,}
                      \AttributeTok{file =}\NormalTok{ here}\SpecialCharTok{::}\FunctionTok{here}\NormalTok{(}\StringTok{"models/fit\_ls\_2\_groups"}\NormalTok{),}
                      \AttributeTok{save\_model =}\NormalTok{ here}\SpecialCharTok{::}\FunctionTok{here}\NormalTok{(}\StringTok{"stancode/fit\_ls\_2\_groups.stan"}\NormalTok{)}
\NormalTok{                      ) }\SpecialCharTok{|\textgreater{}}
  \FunctionTok{add\_criterion}\NormalTok{(}\StringTok{"loo"}\NormalTok{)}
\end{Highlighting}
\end{Shaded}

\hypertarget{groups-1}{%
\subsection{3 groups}\label{groups-1}}

\begin{Shaded}
\begin{Highlighting}[]
\NormalTok{priors }\OtherTok{\textless{}{-}} \FunctionTok{prior}\NormalTok{(}\FunctionTok{normal}\NormalTok{(}\DecValTok{0}\NormalTok{, }\DecValTok{1}\NormalTok{), }\AttributeTok{class =}\NormalTok{ Intercept) }\SpecialCharTok{+}
  \FunctionTok{prior}\NormalTok{(}\FunctionTok{normal}\NormalTok{(}\DecValTok{0}\NormalTok{, }\DecValTok{1}\NormalTok{), }\AttributeTok{class =}\NormalTok{ b) }\SpecialCharTok{+}
  \FunctionTok{prior}\NormalTok{(}\FunctionTok{student\_t}\NormalTok{(}\DecValTok{3}\NormalTok{, }\DecValTok{0}\NormalTok{, }\DecValTok{1}\NormalTok{), }\AttributeTok{class =}\NormalTok{ sd, }\AttributeTok{group =}\NormalTok{ ID) }

\NormalTok{fit\_ls\_3\_groups }\OtherTok{\textless{}{-}} \FunctionTok{brm}\NormalTok{(score }\SpecialCharTok{\textasciitilde{}}\NormalTok{ group3 }\SpecialCharTok{*}\NormalTok{ test }\SpecialCharTok{+}\NormalTok{ (}\DecValTok{1} \SpecialCharTok{|}\NormalTok{ ID),}
                      \AttributeTok{family =}\NormalTok{ gaussian,}
                      \AttributeTok{prior =}\NormalTok{ priors,}
                      \AttributeTok{data =}\NormalTok{ sdtdata\_3,}
                      \AttributeTok{chains =} \DecValTok{4}\NormalTok{, }\AttributeTok{iter =} \DecValTok{2000}\NormalTok{, }\AttributeTok{cores =} \DecValTok{4}\NormalTok{,}
                      \AttributeTok{backend =} \StringTok{"cmdstanr"}\NormalTok{,}
                      \AttributeTok{file =}\NormalTok{ here}\SpecialCharTok{::}\FunctionTok{here}\NormalTok{(}\StringTok{"models/fit\_ls\_3\_groups"}\NormalTok{),}
                      \AttributeTok{save\_model =}\NormalTok{ here}\SpecialCharTok{::}\FunctionTok{here}\NormalTok{(}\StringTok{"stancode/fit\_ls\_3\_groups.stan"}\NormalTok{)}
\NormalTok{                      ) }\SpecialCharTok{|\textgreater{}}
  \FunctionTok{add\_criterion}\NormalTok{(}\StringTok{"loo"}\NormalTok{)}
\end{Highlighting}
\end{Shaded}

\hypertarget{groups-controls-musicians-synaesthetes}{%
\subsection{3 groups (controls, musicians, synaesthetes)}\label{groups-controls-musicians-synaesthetes}}

\begin{Shaded}
\begin{Highlighting}[]
\NormalTok{priors }\OtherTok{\textless{}{-}} \FunctionTok{prior}\NormalTok{(}\FunctionTok{normal}\NormalTok{(}\DecValTok{0}\NormalTok{, }\DecValTok{1}\NormalTok{), }\AttributeTok{class =}\NormalTok{ Intercept) }\SpecialCharTok{+}
  \FunctionTok{prior}\NormalTok{(}\FunctionTok{normal}\NormalTok{(}\DecValTok{0}\NormalTok{, }\DecValTok{1}\NormalTok{), }\AttributeTok{class =}\NormalTok{ b) }\SpecialCharTok{+}
  \FunctionTok{prior}\NormalTok{(}\FunctionTok{student\_t}\NormalTok{(}\DecValTok{3}\NormalTok{, }\DecValTok{0}\NormalTok{, }\DecValTok{1}\NormalTok{), }\AttributeTok{class =}\NormalTok{ sd, }\AttributeTok{group =}\NormalTok{ ID) }

\NormalTok{fit\_ls\_3\_groups\_alt }\OtherTok{\textless{}{-}} \FunctionTok{brm}\NormalTok{(score }\SpecialCharTok{\textasciitilde{}}\NormalTok{ group3alt }\SpecialCharTok{*}\NormalTok{ test }\SpecialCharTok{+}\NormalTok{ (}\DecValTok{1} \SpecialCharTok{|}\NormalTok{ ID),}
                      \AttributeTok{family =}\NormalTok{ gaussian,}
                      \AttributeTok{prior =}\NormalTok{ priors,}
                      \AttributeTok{data =}\NormalTok{ sdtdata\_3alt,}
                      \AttributeTok{chains =} \DecValTok{4}\NormalTok{, }\AttributeTok{iter =} \DecValTok{2000}\NormalTok{, }\AttributeTok{cores =} \DecValTok{4}\NormalTok{,}
                      \AttributeTok{backend =} \StringTok{"cmdstanr"}\NormalTok{,}
                      \AttributeTok{file =}\NormalTok{ here}\SpecialCharTok{::}\FunctionTok{here}\NormalTok{(}\StringTok{"models/fit\_ls\_3\_groups\_alt"}\NormalTok{),}
                      \AttributeTok{save\_model =}\NormalTok{ here}\SpecialCharTok{::}\FunctionTok{here}\NormalTok{(}\StringTok{"stancode/fit\_ls\_3\_groups\_alt.stan"}\NormalTok{)}
\NormalTok{                      ) }\SpecialCharTok{|\textgreater{}}
  \FunctionTok{add\_criterion}\NormalTok{(}\StringTok{"loo"}\NormalTok{)}
\end{Highlighting}
\end{Shaded}

\hypertarget{groups-2}{%
\subsection{4 groups}\label{groups-2}}

\begin{Shaded}
\begin{Highlighting}[]
\NormalTok{priors }\OtherTok{\textless{}{-}} \FunctionTok{prior}\NormalTok{(}\FunctionTok{normal}\NormalTok{(}\DecValTok{0}\NormalTok{, }\DecValTok{1}\NormalTok{), }\AttributeTok{class =}\NormalTok{ Intercept) }\SpecialCharTok{+}
  \FunctionTok{prior}\NormalTok{(}\FunctionTok{normal}\NormalTok{(}\DecValTok{0}\NormalTok{, }\DecValTok{1}\NormalTok{), }\AttributeTok{class =}\NormalTok{ b) }\SpecialCharTok{+}
  \FunctionTok{prior}\NormalTok{(}\FunctionTok{student\_t}\NormalTok{(}\DecValTok{3}\NormalTok{, }\DecValTok{0}\NormalTok{, }\DecValTok{1}\NormalTok{), }\AttributeTok{class =}\NormalTok{ sd, }\AttributeTok{group =}\NormalTok{ ID) }

\NormalTok{fit\_ls\_4\_groups }\OtherTok{\textless{}{-}} \FunctionTok{brm}\NormalTok{(score }\SpecialCharTok{\textasciitilde{}}\NormalTok{ group4 }\SpecialCharTok{*}\NormalTok{ test }\SpecialCharTok{+}\NormalTok{ (}\DecValTok{1} \SpecialCharTok{|}\NormalTok{ ID),}
                      \AttributeTok{family =}\NormalTok{ gaussian,}
                      \AttributeTok{prior =}\NormalTok{ priors,}
                      \AttributeTok{data =}\NormalTok{ dd,}
                      \AttributeTok{chains =} \DecValTok{4}\NormalTok{, }\AttributeTok{iter =} \DecValTok{2000}\NormalTok{, }\AttributeTok{cores =} \DecValTok{4}\NormalTok{,}
                      \AttributeTok{backend =} \StringTok{"cmdstanr"}\NormalTok{,}
                      \AttributeTok{file =}\NormalTok{ here}\SpecialCharTok{::}\FunctionTok{here}\NormalTok{(}\StringTok{"models/fit\_ls\_4\_groups"}\NormalTok{),}
                      \AttributeTok{save\_model =}\NormalTok{ here}\SpecialCharTok{::}\FunctionTok{here}\NormalTok{(}\StringTok{"stancode/fit\_ls\_4\_groups.stan"}\NormalTok{)}
\NormalTok{                      ) }\SpecialCharTok{|\textgreater{}}
  \FunctionTok{add\_criterion}\NormalTok{(}\StringTok{"loo"}\NormalTok{)}
\end{Highlighting}
\end{Shaded}

\begin{Shaded}
\begin{Highlighting}[]
\FunctionTok{mcmc\_plot}\NormalTok{(fit\_ls\_4\_groups)}
\end{Highlighting}
\end{Shaded}

\includegraphics{supplementary_files/figure-latex/unnamed-chunk-17-1.pdf}

\hypertarget{model-comparison}{%
\subsection{Model comparison}\label{model-comparison}}

The models cannot be distringuished based on their out-of-sample predictive accuracy (loo). In order to perform meaningful model comparisons, i.e.~hypothesis tests, we need a different approach, possibly based on posterior predictive checks.

\begin{Shaded}
\begin{Highlighting}[]
\FunctionTok{loo\_compare}\NormalTok{(fit\_ls\_2\_groups, }
\NormalTok{            fit\_ls\_3\_groups, }
\NormalTok{            fit\_ls\_3\_groups\_alt, }
\NormalTok{            fit\_ls\_4\_groups)}
\end{Highlighting}
\end{Shaded}

\begin{verbatim}
##                     elpd_diff se_diff
## fit_ls_2_groups      0.0       0.0   
## fit_ls_3_groups_alt -1.6       1.7   
## fit_ls_3_groups     -1.8       1.3   
## fit_ls_4_groups     -4.4       2.0
\end{verbatim}

\hypertarget{contrasts}{%
\subsection{Contrasts}\label{contrasts}}

\begin{Shaded}
\begin{Highlighting}[]
\NormalTok{epred\_ls\_4\_groups }\OtherTok{\textless{}{-}}\NormalTok{ dd }\SpecialCharTok{|\textgreater{}}
  \FunctionTok{data\_grid}\NormalTok{(group4, test) }\SpecialCharTok{|\textgreater{}}
  \FunctionTok{add\_epred\_draws}\NormalTok{(fit\_ls\_4\_groups,}
                  \AttributeTok{re\_formula =} \SpecialCharTok{\textasciitilde{}}\NormalTok{ID,}
                  \AttributeTok{ndraws =} \DecValTok{500}\NormalTok{)}
\end{Highlighting}
\end{Shaded}

\begin{Shaded}
\begin{Highlighting}[]
\NormalTok{epred\_ls\_4\_groups }\SpecialCharTok{|\textgreater{}}
  \FunctionTok{ggplot}\NormalTok{(}\FunctionTok{aes}\NormalTok{(}\AttributeTok{x =}\NormalTok{ test, }\AttributeTok{y =}\NormalTok{ .epred, }\AttributeTok{color =}\NormalTok{ group4)) }\SpecialCharTok{+}
  \FunctionTok{stat\_pointinterval}\NormalTok{(}\AttributeTok{position =} \FunctionTok{position\_dodge}\NormalTok{(}\AttributeTok{width =}\NormalTok{ .}\DecValTok{4}\NormalTok{)) }\SpecialCharTok{+}
  \FunctionTok{scale\_size\_continuous}\NormalTok{(}\AttributeTok{guide =} \ConstantTok{FALSE}\NormalTok{) }\SpecialCharTok{+}
  \FunctionTok{scale\_color\_viridis\_d}\NormalTok{(}\AttributeTok{begin =} \FloatTok{0.0}\NormalTok{, }\AttributeTok{end =} \FloatTok{0.8}\NormalTok{) }\SpecialCharTok{+}
  \FunctionTok{theme\_clean}\NormalTok{()}
\end{Highlighting}
\end{Shaded}

\begin{verbatim}
## Warning: The `guide` argument in `scale_*()` cannot be `FALSE`. This was deprecated in
## ggplot2 3.3.4.
## i Please use "none" instead.
\end{verbatim}

\includegraphics{supplementary_files/figure-latex/unnamed-chunk-20-1.pdf}

The above figure shoes the expectations of the posterior predictive distributions for each group in each of the three test conditions. THe synaesthesia group has a consistently higher expected learning score over all tests. In this case, however, we are interested in the comparison between musicians with relative and absolute pitch for all three tests.

\begin{Shaded}
\begin{Highlighting}[]
\NormalTok{epred\_ls\_4\_groups\_contrast }\OtherTok{\textless{}{-}} 
  \FunctionTok{expand\_grid}\NormalTok{(}\AttributeTok{group4 =} \FunctionTok{c}\NormalTok{(}\StringTok{"relpitch"}\NormalTok{, }\StringTok{"abspitch"}\NormalTok{), }
         \AttributeTok{test =} \FunctionTok{levels}\NormalTok{(dd}\SpecialCharTok{$}\NormalTok{test)) }\SpecialCharTok{|\textgreater{}} 
  \FunctionTok{add\_epred\_draws}\NormalTok{(fit\_ls\_4\_groups,}
                  \AttributeTok{re\_formula =} \SpecialCharTok{\textasciitilde{}}\NormalTok{ID,}
                  \AttributeTok{ndraws =} \DecValTok{500}\NormalTok{)}
\end{Highlighting}
\end{Shaded}

\begin{Shaded}
\begin{Highlighting}[]
\NormalTok{epred\_ls\_4\_groups\_contrast }\SpecialCharTok{|\textgreater{}} 
    \FunctionTok{ggplot}\NormalTok{(}\FunctionTok{aes}\NormalTok{(}\AttributeTok{x =}\NormalTok{ test, }\AttributeTok{y =}\NormalTok{ .epred, }\AttributeTok{color =}\NormalTok{ group4)) }\SpecialCharTok{+}
  \FunctionTok{stat\_pointinterval}\NormalTok{(}\AttributeTok{position =} \FunctionTok{position\_dodge}\NormalTok{(}\AttributeTok{width =}\NormalTok{ .}\DecValTok{4}\NormalTok{)) }\SpecialCharTok{+}
  \FunctionTok{scale\_size\_continuous}\NormalTok{(}\AttributeTok{guide =} \ConstantTok{FALSE}\NormalTok{) }\SpecialCharTok{+}
  \FunctionTok{scale\_color\_viridis\_d}\NormalTok{(}\AttributeTok{begin =} \FloatTok{0.0}\NormalTok{, }\AttributeTok{end =} \FloatTok{0.8}\NormalTok{) }\SpecialCharTok{+}
  \FunctionTok{theme\_clean}\NormalTok{()}
\end{Highlighting}
\end{Shaded}

\includegraphics{supplementary_files/figure-latex/unnamed-chunk-22-1.pdf}

\begin{Shaded}
\begin{Highlighting}[]
\NormalTok{contrasts\_rel\_abs }\OtherTok{\textless{}{-}}\NormalTok{ fit\_ls\_4\_groups }\SpecialCharTok{|\textgreater{}} 
\NormalTok{  emmeans}\SpecialCharTok{::}\FunctionTok{emmeans}\NormalTok{(}\SpecialCharTok{\textasciitilde{}}\NormalTok{group4 }\SpecialCharTok{|}\NormalTok{ test) }\SpecialCharTok{|\textgreater{}} 
\NormalTok{  emmeans}\SpecialCharTok{::}\FunctionTok{contrast}\NormalTok{(}\StringTok{"pairwise"}\NormalTok{) }\SpecialCharTok{|\textgreater{}} 
  \FunctionTok{gather\_emmeans\_draws}\NormalTok{() }\SpecialCharTok{|\textgreater{}} 
  \FunctionTok{mutate}\NormalTok{(}\AttributeTok{indicator =} \FunctionTok{if\_else}\NormalTok{(contrast }\SpecialCharTok{==} \StringTok{"relpitch {-} abspitch"}\NormalTok{, }\DecValTok{1}\NormalTok{, }\DecValTok{0}\NormalTok{))}
\end{Highlighting}
\end{Shaded}

\begin{Shaded}
\begin{Highlighting}[]
\NormalTok{zero\_color }\OtherTok{\textless{}{-}}\NormalTok{ ggokabeito}\SpecialCharTok{::}\FunctionTok{palette\_okabe\_ito}\NormalTok{()[}\DecValTok{1}\NormalTok{]}

\NormalTok{contrasts\_rel\_abs }\SpecialCharTok{|\textgreater{}} 
  \FunctionTok{ggplot}\NormalTok{(}\FunctionTok{aes}\NormalTok{(}\AttributeTok{y =}\NormalTok{ contrast, }\AttributeTok{x =}\NormalTok{ .value, }\AttributeTok{fill =} \FunctionTok{as\_factor}\NormalTok{(indicator))) }\SpecialCharTok{+}
  \FunctionTok{geom\_vline}\NormalTok{(}\AttributeTok{xintercept =} \DecValTok{0}\NormalTok{, }\AttributeTok{color =} \StringTok{"black"}\NormalTok{, }\AttributeTok{linetype =} \StringTok{"dashed"}\NormalTok{) }\SpecialCharTok{+}
  \FunctionTok{geom\_eye}\NormalTok{() }\SpecialCharTok{+}
  \FunctionTok{scale\_fill\_okabe\_ito}\NormalTok{(}\AttributeTok{guide=}\StringTok{"none"}\NormalTok{) }\SpecialCharTok{+}
  \CommentTok{\# stat\_summary(aes(group = NA), fun.y = mean, geom = "line") +}
  \FunctionTok{facet\_grid}\NormalTok{(}\SpecialCharTok{\textasciitilde{}}\NormalTok{ test) }\SpecialCharTok{+}
  \FunctionTok{theme\_clean}\NormalTok{()}
\end{Highlighting}
\end{Shaded}

\begin{verbatim}
## Warning: 'geom_eye' is deprecated.
## Use 'stat_eye' instead.
## See help("Deprecated") and help("tidybayes-deprecated").
\end{verbatim}

\includegraphics{supplementary_files/figure-latex/unnamed-chunk-24-1.pdf}

The above figure shows the posterior distributions of the pairwise group differences for each test, in the model allowing for 4 separate groups. When allowing for a separate group for musicians with synaesthesia, the expected difference between musicians with relative and absolute pitch is consistenly centred on zero for all testing conditions (blue distributions).

\hypertarget{individial-responses}{%
\section{Individial responses}\label{individial-responses}}

The learning score is computed by averaging over individual ratings. We thus lose information about the subject- and group specific usage of individual ratings. Of particular interest are the extreme rating categories, which denote extreme certainty that either \emph{new} items have never been seen (rating 1) or that \emph{old} items have been previously seen (rating 5).

From visual inspection of individual subjects' ratings it is apparent that synaesthetes tend to use the extreme categories more often; in particular, they more frequently reject previously unseen items with a rating of \(1\).

\begin{Shaded}
\begin{Highlighting}[]
\NormalTok{d }\SpecialCharTok{|\textgreater{}}
  \FunctionTok{filter}\NormalTok{(group4 }\SpecialCharTok{==} \StringTok{"syn"}\NormalTok{) }\SpecialCharTok{|\textgreater{}}
  \FunctionTok{mutate}\NormalTok{(}\AttributeTok{ID =} \FunctionTok{fct\_drop}\NormalTok{(ID)) }\SpecialCharTok{|\textgreater{}}
  \FunctionTok{ggplot}\NormalTok{(}\FunctionTok{aes}\NormalTok{(}\AttributeTok{x =}\NormalTok{ oldnew, }\AttributeTok{fill =}\NormalTok{ rating)) }\SpecialCharTok{+}
  \FunctionTok{geom\_bar}\NormalTok{(}\AttributeTok{position =} \FunctionTok{position\_fill}\NormalTok{(}\AttributeTok{reverse =} \ConstantTok{TRUE}\NormalTok{)) }\SpecialCharTok{+}
  \FunctionTok{scale\_y\_continuous}\NormalTok{(}\AttributeTok{labels =}\NormalTok{ scales}\SpecialCharTok{::}\NormalTok{percent) }\SpecialCharTok{+}
  \FunctionTok{facet\_wrap}\NormalTok{(}\SpecialCharTok{\textasciitilde{}}\NormalTok{ID) }\SpecialCharTok{+}
  \FunctionTok{ylab}\NormalTok{(}\StringTok{"Proportion"}\NormalTok{) }\SpecialCharTok{+}
  \FunctionTok{labs}\NormalTok{(}\AttributeTok{x =} \StringTok{"Triplet"}\NormalTok{, }\AttributeTok{y =} \StringTok{"Proportion"}\NormalTok{, }\AttributeTok{color =} \StringTok{"Rating"}\NormalTok{) }\SpecialCharTok{+}
  \FunctionTok{theme\_clean}\NormalTok{() }\SpecialCharTok{+}
  \FunctionTok{ggtitle}\NormalTok{(}\StringTok{"Synaesthetes"}\NormalTok{)}
\end{Highlighting}
\end{Shaded}

\includegraphics{supplementary_files/figure-latex/unnamed-chunk-25-1.pdf}

\begin{Shaded}
\begin{Highlighting}[]
\NormalTok{d }\SpecialCharTok{|\textgreater{}}
  \FunctionTok{filter}\NormalTok{(group4 }\SpecialCharTok{==} \StringTok{"abspitch"}\NormalTok{) }\SpecialCharTok{|\textgreater{}}
  \FunctionTok{mutate}\NormalTok{(}\AttributeTok{ID =} \FunctionTok{fct\_drop}\NormalTok{(ID)) }\SpecialCharTok{|\textgreater{}}
  \FunctionTok{ggplot}\NormalTok{(}\FunctionTok{aes}\NormalTok{(}\AttributeTok{x =}\NormalTok{ oldnew, }\AttributeTok{fill =}\NormalTok{ rating)) }\SpecialCharTok{+}
  \FunctionTok{geom\_bar}\NormalTok{(}\AttributeTok{position =} \FunctionTok{position\_fill}\NormalTok{(}\AttributeTok{reverse =} \ConstantTok{TRUE}\NormalTok{)) }\SpecialCharTok{+}
  \FunctionTok{scale\_y\_continuous}\NormalTok{(}\AttributeTok{labels =}\NormalTok{ scales}\SpecialCharTok{::}\NormalTok{percent) }\SpecialCharTok{+}
  \FunctionTok{facet\_wrap}\NormalTok{(}\SpecialCharTok{\textasciitilde{}}\NormalTok{ID) }\SpecialCharTok{+}
  \FunctionTok{ylab}\NormalTok{(}\StringTok{"Proportion"}\NormalTok{) }\SpecialCharTok{+}
  \FunctionTok{labs}\NormalTok{(}\AttributeTok{x =} \StringTok{"Triplet"}\NormalTok{, }\AttributeTok{y =} \StringTok{"Proportion"}\NormalTok{, }\AttributeTok{color =} \StringTok{"Rating"}\NormalTok{) }\SpecialCharTok{+}
  \FunctionTok{theme\_clean}\NormalTok{() }\SpecialCharTok{+}
  \FunctionTok{ggtitle}\NormalTok{(}\StringTok{"Absolute Pitch"}\NormalTok{)}
\end{Highlighting}
\end{Shaded}

\includegraphics{supplementary_files/figure-latex/unnamed-chunk-26-1.pdf}

\begin{Shaded}
\begin{Highlighting}[]
\NormalTok{d }\SpecialCharTok{|\textgreater{}}
  \FunctionTok{filter}\NormalTok{(group4 }\SpecialCharTok{==} \StringTok{"relpitch"}\NormalTok{) }\SpecialCharTok{|\textgreater{}}
  \FunctionTok{mutate}\NormalTok{(}\AttributeTok{ID =} \FunctionTok{fct\_drop}\NormalTok{(ID)) }\SpecialCharTok{|\textgreater{}}
  \FunctionTok{ggplot}\NormalTok{(}\FunctionTok{aes}\NormalTok{(}\AttributeTok{x =}\NormalTok{ oldnew, }\AttributeTok{fill =}\NormalTok{ rating)) }\SpecialCharTok{+}
  \FunctionTok{geom\_bar}\NormalTok{(}\AttributeTok{position =} \FunctionTok{position\_fill}\NormalTok{(}\AttributeTok{reverse =} \ConstantTok{TRUE}\NormalTok{)) }\SpecialCharTok{+}
  \FunctionTok{scale\_y\_continuous}\NormalTok{(}\AttributeTok{labels =}\NormalTok{ scales}\SpecialCharTok{::}\NormalTok{percent) }\SpecialCharTok{+}
  \CommentTok{\# scale\_x\_continuous(breaks = 1:5) +}
  \FunctionTok{facet\_wrap}\NormalTok{(}\SpecialCharTok{\textasciitilde{}}\NormalTok{ID) }\SpecialCharTok{+}
  \FunctionTok{ylab}\NormalTok{(}\StringTok{"Proportion"}\NormalTok{) }\SpecialCharTok{+}
  \FunctionTok{labs}\NormalTok{(}\AttributeTok{x =} \StringTok{"Triplet"}\NormalTok{, }\AttributeTok{y =} \StringTok{"Proportion"}\NormalTok{, }\AttributeTok{color =} \StringTok{"Rating"}\NormalTok{) }\SpecialCharTok{+}
  \FunctionTok{theme\_clean}\NormalTok{() }\SpecialCharTok{+}
  \FunctionTok{ggtitle}\NormalTok{(}\StringTok{"Relative Pitch"}\NormalTok{)}
\end{Highlighting}
\end{Shaded}

\includegraphics{supplementary_files/figure-latex/unnamed-chunk-27-1.pdf}

\begin{Shaded}
\begin{Highlighting}[]
\NormalTok{d }\SpecialCharTok{|\textgreater{}}
  \FunctionTok{filter}\NormalTok{(group4 }\SpecialCharTok{==} \StringTok{"control"}\NormalTok{) }\SpecialCharTok{|\textgreater{}}
  \FunctionTok{mutate}\NormalTok{(}\AttributeTok{ID =} \FunctionTok{fct\_drop}\NormalTok{(ID)) }\SpecialCharTok{|\textgreater{}}
  \FunctionTok{ggplot}\NormalTok{(}\FunctionTok{aes}\NormalTok{(}\AttributeTok{x =}\NormalTok{ oldnew, }\AttributeTok{fill =}\NormalTok{ rating)) }\SpecialCharTok{+}
  \FunctionTok{geom\_bar}\NormalTok{(}\AttributeTok{position =} \FunctionTok{position\_fill}\NormalTok{(}\AttributeTok{reverse =} \ConstantTok{TRUE}\NormalTok{)) }\SpecialCharTok{+}
  \FunctionTok{scale\_y\_continuous}\NormalTok{(}\AttributeTok{labels =}\NormalTok{ scales}\SpecialCharTok{::}\NormalTok{percent) }\SpecialCharTok{+}
  \FunctionTok{facet\_wrap}\NormalTok{(}\SpecialCharTok{\textasciitilde{}}\NormalTok{ID) }\SpecialCharTok{+}
  \FunctionTok{ylab}\NormalTok{(}\StringTok{"Proportion"}\NormalTok{) }\SpecialCharTok{+}
  \FunctionTok{ggtitle}\NormalTok{(}\StringTok{"Control"}\NormalTok{)}
\end{Highlighting}
\end{Shaded}

\includegraphics{supplementary_files/figure-latex/unnamed-chunk-28-1.pdf}

\hypertarget{ordinal-regression-models}{%
\section{Ordinal regression models}\label{ordinal-regression-models}}

\hypertarget{theory}{%
\subsection{Theory}\label{theory}}

\begin{itemize}
\tightlist
\item
  Although ordinal data are not metric, they are often analyzed using methods that assume metric responses. This practice may lead to serious errors in inference (Liddell \& Kruschke, 2018).
\item
  Ordinal variables: categories have an ordering, but it is unknown

  \begin{itemize}
  \tightlist
  \item
    what the \textbf{psychological distance} between them is
  \item
    whether distances between categories are the same across participants
  \end{itemize}
\end{itemize}

\hypertarget{ordinal-regression}{%
\subsubsection{Ordinal regression}\label{ordinal-regression}}

\begin{itemize}
\tightlist
\item
  Use the framework of signal detection (unequal variance SDT or logistic model with heteroscedastic error)
\item
  Work with raw responses, instead of summarizing data
\item
  Quantify uncertainty at all levels
\item
  Allows multilevel model (shrinkage could be especially important due to low number of subjects)
\end{itemize}

\hypertarget{unequal-variance-logistic-sdt-model}{%
\subsubsection{Unequal Variance (logistic) SDT Model}\label{unequal-variance-logistic-sdt-model}}

\begin{itemize}
\tightlist
\item
  Item is either old or new
\item
  Subjects do not provide binary old or new responses, but instead give their responses on a 5-point rating scale
\item
  Subjects rate their confidence in whether the item was old or new (actually, how frequently the item was presented)
\item
  Subjects set a number of criteria for the ratings, such that greater evidence is required for 5-responses, than 4-responses, for example.
\end{itemize}

\[ P(Y \leq k | X) = F\left(  \frac{ c_k - dX} { \sigma_X }  \right)\]

for \(k=1\) to \(K-1\), where

\begin{itemize}
\tightlist
\item
  \(K\) is the number of response categories
\item
  \(Y\) is a response rating, taking on the values \(k=1\) to \(K\)
\item
  \(F\) is a cumulative distribution function
\item
  \(c_k\) are response criteria
\item
  \(\sigma_X\) is the standard deviation of the latent distribution
\end{itemize}

\begin{verbatim}
## Warning: Using `size` aesthetic for lines was deprecated in ggplot2 3.4.0.
## i Please use `linewidth` instead.
\end{verbatim}

\includegraphics{supplementary_files/figure-latex/unnamed-chunk-29-1.pdf}

The idea is that each individual sets thresholds on the latent scale, depending on the type of the test. We allow the variance of the internal representation to differ between old and new items. Thresholds are shown above as dashed lines. If the internal representation lies between thresholds \(\tau_k\) and \(\tau_{k+1}\), the corresponding rating is chosen.

\hypertarget{brms-models}{%
\subsection{BRMS Models}\label{brms-models}}

Set initial values for sampling the thresholds, assuming initially that the thresholds are evenly distributed, i.e.~each response category has the same probability of being chosen.

\begin{Shaded}
\begin{Highlighting}[]
\FunctionTok{tibble}\NormalTok{(}\AttributeTok{rating =} \DecValTok{1}\SpecialCharTok{:}\DecValTok{5}\NormalTok{) }\SpecialCharTok{|\textgreater{}}
  \FunctionTok{mutate}\NormalTok{(}\AttributeTok{proportion =} \DecValTok{1} \SpecialCharTok{/} \DecValTok{5}\NormalTok{) }\SpecialCharTok{|\textgreater{}}
  \FunctionTok{mutate}\NormalTok{(}\AttributeTok{cumulative\_proportion =} \FunctionTok{cumsum}\NormalTok{(proportion)) }\SpecialCharTok{|\textgreater{}}
  \FunctionTok{mutate}\NormalTok{(}
    \AttributeTok{right\_hand\_threshold =} \FunctionTok{qnorm}\NormalTok{(cumulative\_proportion),}
    \AttributeTok{right\_hand\_threshold\_logit =} \FunctionTok{qlogis}\NormalTok{(cumulative\_proportion)}
\NormalTok{  )}
\end{Highlighting}
\end{Shaded}

\begin{verbatim}
## # A tibble: 5 x 5
##   rating proportion cumulative_proportion right_hand_threshold right_hand_thre~1
##    <int>      <dbl>                 <dbl>                <dbl>             <dbl>
## 1      1        0.2                   0.2               -0.842            -1.39 
## 2      2        0.2                   0.4               -0.253            -0.405
## 3      3        0.2                   0.6                0.253             0.405
## 4      4        0.2                   0.8                0.842             1.39 
## 5      5        0.2                   1                Inf               Inf    
## # ... with abbreviated variable name 1: right_hand_threshold_logit
\end{verbatim}

\hypertarget{groups-3}{%
\subsubsection{2 groups}\label{groups-3}}

\begin{Shaded}
\begin{Highlighting}[]
\NormalTok{priors }\OtherTok{\textless{}{-}} \FunctionTok{prior}\NormalTok{(}\FunctionTok{normal}\NormalTok{(}\SpecialCharTok{{-}}\FloatTok{1.39}\NormalTok{, }\DecValTok{1}\NormalTok{), }\AttributeTok{class =}\NormalTok{ Intercept, }\AttributeTok{coef =} \DecValTok{1}\NormalTok{) }\SpecialCharTok{+}
  \FunctionTok{prior}\NormalTok{(}\FunctionTok{normal}\NormalTok{(}\SpecialCharTok{{-}}\FloatTok{0.405}\NormalTok{, }\DecValTok{1}\NormalTok{), }\AttributeTok{class =}\NormalTok{ Intercept, }\AttributeTok{coef =} \DecValTok{2}\NormalTok{) }\SpecialCharTok{+}
  \FunctionTok{prior}\NormalTok{(}\FunctionTok{normal}\NormalTok{(}\FloatTok{0.405}\NormalTok{, }\DecValTok{1}\NormalTok{), }\AttributeTok{class =}\NormalTok{ Intercept, }\AttributeTok{coef =} \DecValTok{3}\NormalTok{) }\SpecialCharTok{+}
  \FunctionTok{prior}\NormalTok{(}\FunctionTok{normal}\NormalTok{(}\FloatTok{1.39}\NormalTok{, }\DecValTok{1}\NormalTok{), }\AttributeTok{class =}\NormalTok{ Intercept, }\AttributeTok{coef =} \DecValTok{4}\NormalTok{) }\SpecialCharTok{+}
  \FunctionTok{prior}\NormalTok{(}\FunctionTok{normal}\NormalTok{(}\DecValTok{0}\NormalTok{, }\DecValTok{1}\NormalTok{), }\AttributeTok{class =}\NormalTok{ b) }\SpecialCharTok{+}
  \FunctionTok{prior}\NormalTok{(}\FunctionTok{normal}\NormalTok{(}\DecValTok{0}\NormalTok{, }\DecValTok{1}\NormalTok{), }\AttributeTok{class =}\NormalTok{ b, }\AttributeTok{dpar =} \StringTok{"disc"}\NormalTok{) }\SpecialCharTok{+}
  \FunctionTok{prior}\NormalTok{(}\FunctionTok{student\_t}\NormalTok{(}\DecValTok{3}\NormalTok{, }\DecValTok{0}\NormalTok{, }\DecValTok{1}\NormalTok{), }\AttributeTok{class =}\NormalTok{ sd, }\AttributeTok{group =}\NormalTok{ ID) }\SpecialCharTok{+}
  \FunctionTok{prior}\NormalTok{(}\FunctionTok{lkj}\NormalTok{(}\DecValTok{2}\NormalTok{), }\AttributeTok{class =}\NormalTok{ cor, }\AttributeTok{group =}\NormalTok{ ID) }\SpecialCharTok{+}
  \FunctionTok{prior}\NormalTok{(}\FunctionTok{student\_t}\NormalTok{(}\DecValTok{3}\NormalTok{, }\DecValTok{0}\NormalTok{, }\DecValTok{1}\NormalTok{), }\AttributeTok{class =}\NormalTok{ sd, }\AttributeTok{group =}\NormalTok{ item)}

\NormalTok{inits }\OtherTok{\textless{}{-}} \FunctionTok{list}\NormalTok{(}\AttributeTok{Intercept =} \FunctionTok{c}\NormalTok{(}\SpecialCharTok{{-}}\FloatTok{1.39}\NormalTok{, }\SpecialCharTok{{-}}\FloatTok{0.405}\NormalTok{, }\FloatTok{0.405}\NormalTok{, }\FloatTok{1.39}\NormalTok{))}

\NormalTok{formula }\OtherTok{\textless{}{-}} \FunctionTok{bf}\NormalTok{(rating }\SpecialCharTok{\textasciitilde{}}\NormalTok{ oldnew }\SpecialCharTok{*}\NormalTok{ group2 }\SpecialCharTok{*}\NormalTok{ test }\SpecialCharTok{+}
\NormalTok{  (}\DecValTok{1} \SpecialCharTok{+}\NormalTok{ oldnew }\SpecialCharTok{|}\NormalTok{ ID) }\SpecialCharTok{+}\NormalTok{ (}\DecValTok{1} \SpecialCharTok{|}\NormalTok{ item)) }\SpecialCharTok{+}
  \FunctionTok{lf}\NormalTok{(disc }\SpecialCharTok{\textasciitilde{}} \DecValTok{0} \SpecialCharTok{+}\NormalTok{ oldnew }\SpecialCharTok{*}\NormalTok{ test }\SpecialCharTok{+}
\NormalTok{    (}\DecValTok{1} \SpecialCharTok{+}\NormalTok{ oldnew }\SpecialCharTok{|}\NormalTok{ ID) }\SpecialCharTok{+}\NormalTok{ (}\DecValTok{1} \SpecialCharTok{|}\NormalTok{ item), }\AttributeTok{cmc =} \ConstantTok{FALSE}\NormalTok{)}

\NormalTok{fit\_2\_groups }\OtherTok{\textless{}{-}} \FunctionTok{brm}\NormalTok{(formula,}
  \AttributeTok{family =} \FunctionTok{cumulative}\NormalTok{(}\StringTok{"logit"}\NormalTok{),}
  \AttributeTok{data =}\NormalTok{ d,}
  \AttributeTok{prior =}\NormalTok{ priors,}
  \AttributeTok{init =} \FunctionTok{rep}\NormalTok{(}\FunctionTok{list}\NormalTok{(inits), }\DecValTok{4}\NormalTok{),}
  \AttributeTok{chains =} \DecValTok{4}\NormalTok{, }\AttributeTok{iter =} \DecValTok{2000}\NormalTok{, }\AttributeTok{cores =} \DecValTok{4}\NormalTok{,}
  \AttributeTok{backend =} \StringTok{"cmdstanr"}\NormalTok{,}
  \AttributeTok{file =}\NormalTok{ here}\SpecialCharTok{::}\FunctionTok{here}\NormalTok{(}\StringTok{"models/fit\_2\_groups"}\NormalTok{),}
  \AttributeTok{save\_model =} \StringTok{"stancode/fit\_2\_groups.stan"}
\NormalTok{) }\SpecialCharTok{|\textgreater{}}
  \FunctionTok{add\_criterion}\NormalTok{(}\StringTok{"loo"}\NormalTok{)}
\end{Highlighting}
\end{Shaded}

\hypertarget{groups-4}{%
\subsubsection{3 Groups}\label{groups-4}}

\begin{Shaded}
\begin{Highlighting}[]
\NormalTok{priors }\OtherTok{\textless{}{-}} \FunctionTok{prior}\NormalTok{(}\FunctionTok{normal}\NormalTok{(}\SpecialCharTok{{-}}\FloatTok{1.39}\NormalTok{, }\DecValTok{1}\NormalTok{), }\AttributeTok{class =}\NormalTok{ Intercept, }\AttributeTok{coef =} \DecValTok{1}\NormalTok{) }\SpecialCharTok{+}
  \FunctionTok{prior}\NormalTok{(}\FunctionTok{normal}\NormalTok{(}\SpecialCharTok{{-}}\FloatTok{0.405}\NormalTok{, }\DecValTok{1}\NormalTok{), }\AttributeTok{class =}\NormalTok{ Intercept, }\AttributeTok{coef =} \DecValTok{2}\NormalTok{) }\SpecialCharTok{+}
  \FunctionTok{prior}\NormalTok{(}\FunctionTok{normal}\NormalTok{(}\FloatTok{0.405}\NormalTok{, }\DecValTok{1}\NormalTok{), }\AttributeTok{class =}\NormalTok{ Intercept, }\AttributeTok{coef =} \DecValTok{3}\NormalTok{) }\SpecialCharTok{+}
  \FunctionTok{prior}\NormalTok{(}\FunctionTok{normal}\NormalTok{(}\FloatTok{1.39}\NormalTok{, }\DecValTok{1}\NormalTok{), }\AttributeTok{class =}\NormalTok{ Intercept, }\AttributeTok{coef =} \DecValTok{4}\NormalTok{) }\SpecialCharTok{+}
  \FunctionTok{prior}\NormalTok{(}\FunctionTok{normal}\NormalTok{(}\DecValTok{0}\NormalTok{, }\DecValTok{1}\NormalTok{), }\AttributeTok{class =}\NormalTok{ b) }\SpecialCharTok{+}
  \FunctionTok{prior}\NormalTok{(}\FunctionTok{normal}\NormalTok{(}\DecValTok{0}\NormalTok{, }\DecValTok{1}\NormalTok{), }\AttributeTok{class =}\NormalTok{ b, }\AttributeTok{dpar =} \StringTok{"disc"}\NormalTok{) }\SpecialCharTok{+}
  \FunctionTok{prior}\NormalTok{(}\FunctionTok{student\_t}\NormalTok{(}\DecValTok{3}\NormalTok{, }\DecValTok{0}\NormalTok{, }\DecValTok{1}\NormalTok{), }\AttributeTok{class =}\NormalTok{ sd, }\AttributeTok{group =}\NormalTok{ ID) }\SpecialCharTok{+}
  \FunctionTok{prior}\NormalTok{(}\FunctionTok{lkj}\NormalTok{(}\DecValTok{2}\NormalTok{), }\AttributeTok{class =}\NormalTok{ cor, }\AttributeTok{group =}\NormalTok{ ID) }\SpecialCharTok{+}
  \FunctionTok{prior}\NormalTok{(}\FunctionTok{student\_t}\NormalTok{(}\DecValTok{3}\NormalTok{, }\DecValTok{0}\NormalTok{, }\DecValTok{1}\NormalTok{), }\AttributeTok{class =}\NormalTok{ sd, }\AttributeTok{group =}\NormalTok{ item)}

\NormalTok{inits }\OtherTok{\textless{}{-}} \FunctionTok{list}\NormalTok{(}\AttributeTok{Intercept =} \FunctionTok{c}\NormalTok{(}\SpecialCharTok{{-}}\FloatTok{1.39}\NormalTok{, }\SpecialCharTok{{-}}\FloatTok{0.405}\NormalTok{, }\FloatTok{0.405}\NormalTok{, }\FloatTok{1.39}\NormalTok{))}

\NormalTok{formula }\OtherTok{\textless{}{-}} \FunctionTok{bf}\NormalTok{(rating }\SpecialCharTok{\textasciitilde{}}\NormalTok{ oldnew }\SpecialCharTok{*}\NormalTok{ group3 }\SpecialCharTok{*}\NormalTok{ test }\SpecialCharTok{+}
\NormalTok{  (}\DecValTok{1} \SpecialCharTok{+}\NormalTok{ oldnew }\SpecialCharTok{|}\NormalTok{ ID) }\SpecialCharTok{+}\NormalTok{ (}\DecValTok{1} \SpecialCharTok{|}\NormalTok{ item)) }\SpecialCharTok{+}
  \FunctionTok{lf}\NormalTok{(disc }\SpecialCharTok{\textasciitilde{}} \DecValTok{0} \SpecialCharTok{+}\NormalTok{ oldnew }\SpecialCharTok{*}\NormalTok{ test }\SpecialCharTok{+}
\NormalTok{    (}\DecValTok{1} \SpecialCharTok{+}\NormalTok{ oldnew }\SpecialCharTok{|}\NormalTok{ ID) }\SpecialCharTok{+}\NormalTok{ (}\DecValTok{1} \SpecialCharTok{|}\NormalTok{ item), }\AttributeTok{cmc =} \ConstantTok{FALSE}\NormalTok{)}

\NormalTok{fit\_3\_groups }\OtherTok{\textless{}{-}} \FunctionTok{brm}\NormalTok{(formula,}
  \AttributeTok{family =} \FunctionTok{cumulative}\NormalTok{(}\StringTok{"logit"}\NormalTok{),}
  \AttributeTok{data =}\NormalTok{ d,}
  \AttributeTok{prior =}\NormalTok{ priors,}
  \AttributeTok{init =} \FunctionTok{rep}\NormalTok{(}\FunctionTok{list}\NormalTok{(inits), }\DecValTok{4}\NormalTok{),}
  \AttributeTok{chains =} \DecValTok{4}\NormalTok{, }\AttributeTok{iter =} \DecValTok{2000}\NormalTok{, }\AttributeTok{cores =} \DecValTok{4}\NormalTok{,}
  \AttributeTok{backend =} \StringTok{"cmdstanr"}\NormalTok{,}
  \AttributeTok{file =}\NormalTok{ here}\SpecialCharTok{::}\FunctionTok{here}\NormalTok{(}\StringTok{"models/fit\_3\_groups"}\NormalTok{),}
  \AttributeTok{save\_model =} \StringTok{"stancode/fit\_3\_groups.stan"}
\NormalTok{) }\SpecialCharTok{|\textgreater{}}
  \FunctionTok{add\_criterion}\NormalTok{(}\StringTok{"loo"}\NormalTok{)}
\end{Highlighting}
\end{Shaded}

\hypertarget{alternative-groups}{%
\paragraph{3 Alternative Groups}\label{alternative-groups}}

\begin{Shaded}
\begin{Highlighting}[]
\NormalTok{priors }\OtherTok{\textless{}{-}} \FunctionTok{prior}\NormalTok{(}\FunctionTok{normal}\NormalTok{(}\SpecialCharTok{{-}}\FloatTok{1.39}\NormalTok{, }\DecValTok{1}\NormalTok{), }\AttributeTok{class =}\NormalTok{ Intercept, }\AttributeTok{coef =} \DecValTok{1}\NormalTok{) }\SpecialCharTok{+}
  \FunctionTok{prior}\NormalTok{(}\FunctionTok{normal}\NormalTok{(}\SpecialCharTok{{-}}\FloatTok{0.405}\NormalTok{, }\DecValTok{1}\NormalTok{), }\AttributeTok{class =}\NormalTok{ Intercept, }\AttributeTok{coef =} \DecValTok{2}\NormalTok{) }\SpecialCharTok{+}
  \FunctionTok{prior}\NormalTok{(}\FunctionTok{normal}\NormalTok{(}\FloatTok{0.405}\NormalTok{, }\DecValTok{1}\NormalTok{), }\AttributeTok{class =}\NormalTok{ Intercept, }\AttributeTok{coef =} \DecValTok{3}\NormalTok{) }\SpecialCharTok{+}
  \FunctionTok{prior}\NormalTok{(}\FunctionTok{normal}\NormalTok{(}\FloatTok{1.39}\NormalTok{, }\DecValTok{1}\NormalTok{), }\AttributeTok{class =}\NormalTok{ Intercept, }\AttributeTok{coef =} \DecValTok{4}\NormalTok{) }\SpecialCharTok{+}
  \FunctionTok{prior}\NormalTok{(}\FunctionTok{normal}\NormalTok{(}\DecValTok{0}\NormalTok{, }\DecValTok{1}\NormalTok{), }\AttributeTok{class =}\NormalTok{ b) }\SpecialCharTok{+}
  \FunctionTok{prior}\NormalTok{(}\FunctionTok{normal}\NormalTok{(}\DecValTok{0}\NormalTok{, }\DecValTok{1}\NormalTok{), }\AttributeTok{class =}\NormalTok{ b, }\AttributeTok{dpar =} \StringTok{"disc"}\NormalTok{) }\SpecialCharTok{+}
  \FunctionTok{prior}\NormalTok{(}\FunctionTok{student\_t}\NormalTok{(}\DecValTok{3}\NormalTok{, }\DecValTok{0}\NormalTok{, }\DecValTok{1}\NormalTok{), }\AttributeTok{class =}\NormalTok{ sd, }\AttributeTok{group =}\NormalTok{ ID) }\SpecialCharTok{+}
  \FunctionTok{prior}\NormalTok{(}\FunctionTok{lkj}\NormalTok{(}\DecValTok{2}\NormalTok{), }\AttributeTok{class =}\NormalTok{ cor, }\AttributeTok{group =}\NormalTok{ ID) }\SpecialCharTok{+}
  \FunctionTok{prior}\NormalTok{(}\FunctionTok{student\_t}\NormalTok{(}\DecValTok{3}\NormalTok{, }\DecValTok{0}\NormalTok{, }\DecValTok{1}\NormalTok{), }\AttributeTok{class =}\NormalTok{ sd, }\AttributeTok{group =}\NormalTok{ item)}


\NormalTok{inits }\OtherTok{\textless{}{-}} \FunctionTok{list}\NormalTok{(}\AttributeTok{Intercept =} \FunctionTok{c}\NormalTok{(}\SpecialCharTok{{-}}\FloatTok{1.39}\NormalTok{, }\SpecialCharTok{{-}}\FloatTok{0.405}\NormalTok{, }\FloatTok{0.405}\NormalTok{, }\FloatTok{1.39}\NormalTok{))}

\NormalTok{formula }\OtherTok{\textless{}{-}} \FunctionTok{bf}\NormalTok{(rating }\SpecialCharTok{\textasciitilde{}}\NormalTok{ oldnew }\SpecialCharTok{*}\NormalTok{ group3alt }\SpecialCharTok{*}\NormalTok{ test }\SpecialCharTok{+}
\NormalTok{  (}\DecValTok{1} \SpecialCharTok{+}\NormalTok{ oldnew }\SpecialCharTok{|}\NormalTok{ ID) }\SpecialCharTok{+}\NormalTok{ (}\DecValTok{1} \SpecialCharTok{|}\NormalTok{ item)) }\SpecialCharTok{+}
  \FunctionTok{lf}\NormalTok{(disc }\SpecialCharTok{\textasciitilde{}} \DecValTok{0} \SpecialCharTok{+}\NormalTok{ oldnew }\SpecialCharTok{*}\NormalTok{ test }\SpecialCharTok{+}
\NormalTok{    (}\DecValTok{1} \SpecialCharTok{+}\NormalTok{ oldnew }\SpecialCharTok{|}\NormalTok{ ID) }\SpecialCharTok{+}\NormalTok{ (}\DecValTok{1} \SpecialCharTok{|}\NormalTok{ item), }\AttributeTok{cmc =} \ConstantTok{FALSE}\NormalTok{)}


\NormalTok{fit\_3\_groups\_alt }\OtherTok{\textless{}{-}} \FunctionTok{brm}\NormalTok{(formula,}
  \AttributeTok{family =} \FunctionTok{cumulative}\NormalTok{(}\StringTok{"logit"}\NormalTok{),}
  \AttributeTok{data =}\NormalTok{ d,}
  \AttributeTok{prior =}\NormalTok{ priors,}
  \AttributeTok{init =} \FunctionTok{rep}\NormalTok{(}\FunctionTok{list}\NormalTok{(inits), }\DecValTok{4}\NormalTok{),}
  \AttributeTok{chains =} \DecValTok{4}\NormalTok{, }\AttributeTok{iter =} \DecValTok{2000}\NormalTok{, }\AttributeTok{cores =} \DecValTok{4}\NormalTok{,}
  \AttributeTok{backend =} \StringTok{"cmdstanr"}\NormalTok{,}
  \AttributeTok{file =}\NormalTok{ here}\SpecialCharTok{::}\FunctionTok{here}\NormalTok{(}\StringTok{"models/fit\_3\_groups\_alt"}\NormalTok{),}
  \AttributeTok{save\_model =} \StringTok{"stancode/fit\_3\_groups\_alt.stan"}
\NormalTok{) }\SpecialCharTok{|\textgreater{}}
  \FunctionTok{add\_criterion}\NormalTok{(}\StringTok{"loo"}\NormalTok{)}
\end{Highlighting}
\end{Shaded}

\hypertarget{groups-5}{%
\subsubsection{4 Groups}\label{groups-5}}

\begin{Shaded}
\begin{Highlighting}[]
\NormalTok{priors }\OtherTok{\textless{}{-}} \FunctionTok{prior}\NormalTok{(}\FunctionTok{normal}\NormalTok{(}\SpecialCharTok{{-}}\FloatTok{1.39}\NormalTok{, }\DecValTok{1}\NormalTok{), }\AttributeTok{class =}\NormalTok{ Intercept, }\AttributeTok{coef =} \DecValTok{1}\NormalTok{) }\SpecialCharTok{+}
  \FunctionTok{prior}\NormalTok{(}\FunctionTok{normal}\NormalTok{(}\SpecialCharTok{{-}}\FloatTok{0.405}\NormalTok{, }\DecValTok{1}\NormalTok{), }\AttributeTok{class =}\NormalTok{ Intercept, }\AttributeTok{coef =} \DecValTok{2}\NormalTok{) }\SpecialCharTok{+}
  \FunctionTok{prior}\NormalTok{(}\FunctionTok{normal}\NormalTok{(}\FloatTok{0.405}\NormalTok{, }\DecValTok{1}\NormalTok{), }\AttributeTok{class =}\NormalTok{ Intercept, }\AttributeTok{coef =} \DecValTok{3}\NormalTok{) }\SpecialCharTok{+}
  \FunctionTok{prior}\NormalTok{(}\FunctionTok{normal}\NormalTok{(}\FloatTok{1.39}\NormalTok{, }\DecValTok{1}\NormalTok{), }\AttributeTok{class =}\NormalTok{ Intercept, }\AttributeTok{coef =} \DecValTok{4}\NormalTok{) }\SpecialCharTok{+}
  \FunctionTok{prior}\NormalTok{(}\FunctionTok{normal}\NormalTok{(}\DecValTok{0}\NormalTok{, }\DecValTok{1}\NormalTok{), }\AttributeTok{class =}\NormalTok{ b) }\SpecialCharTok{+}
  \FunctionTok{prior}\NormalTok{(}\FunctionTok{normal}\NormalTok{(}\DecValTok{0}\NormalTok{, }\DecValTok{1}\NormalTok{), }\AttributeTok{class =}\NormalTok{ b, }\AttributeTok{dpar =} \StringTok{"disc"}\NormalTok{) }\SpecialCharTok{+}
  \FunctionTok{prior}\NormalTok{(}\FunctionTok{student\_t}\NormalTok{(}\DecValTok{3}\NormalTok{, }\DecValTok{0}\NormalTok{, }\DecValTok{1}\NormalTok{), }\AttributeTok{class =}\NormalTok{ sd, }\AttributeTok{group =}\NormalTok{ ID) }\SpecialCharTok{+}
  \FunctionTok{prior}\NormalTok{(}\FunctionTok{lkj}\NormalTok{(}\DecValTok{2}\NormalTok{), }\AttributeTok{class =}\NormalTok{ cor, }\AttributeTok{group =}\NormalTok{ ID) }\SpecialCharTok{+}
  \FunctionTok{prior}\NormalTok{(}\FunctionTok{student\_t}\NormalTok{(}\DecValTok{3}\NormalTok{, }\DecValTok{0}\NormalTok{, }\DecValTok{1}\NormalTok{), }\AttributeTok{class =}\NormalTok{ sd, }\AttributeTok{group =}\NormalTok{ item)}


\NormalTok{inits }\OtherTok{\textless{}{-}} \FunctionTok{list}\NormalTok{(}\AttributeTok{Intercept =} \FunctionTok{c}\NormalTok{(}\SpecialCharTok{{-}}\FloatTok{1.39}\NormalTok{, }\SpecialCharTok{{-}}\FloatTok{0.405}\NormalTok{, }\FloatTok{0.405}\NormalTok{, }\FloatTok{1.39}\NormalTok{))}

\NormalTok{formula }\OtherTok{\textless{}{-}} \FunctionTok{bf}\NormalTok{(rating }\SpecialCharTok{\textasciitilde{}}\NormalTok{ oldnew }\SpecialCharTok{*}\NormalTok{ group4 }\SpecialCharTok{*}\NormalTok{ test }\SpecialCharTok{+}
\NormalTok{  (}\DecValTok{1} \SpecialCharTok{+}\NormalTok{ oldnew }\SpecialCharTok{|}\NormalTok{ ID) }\SpecialCharTok{+}\NormalTok{ (}\DecValTok{1} \SpecialCharTok{|}\NormalTok{ item)) }\SpecialCharTok{+}
  \FunctionTok{lf}\NormalTok{(disc }\SpecialCharTok{\textasciitilde{}} \DecValTok{0} \SpecialCharTok{+}\NormalTok{ oldnew }\SpecialCharTok{*}\NormalTok{ test }\SpecialCharTok{+}
\NormalTok{    (}\DecValTok{1} \SpecialCharTok{+}\NormalTok{ oldnew }\SpecialCharTok{|}\NormalTok{ ID) }\SpecialCharTok{+}\NormalTok{ (}\DecValTok{1} \SpecialCharTok{|}\NormalTok{ item), }\AttributeTok{cmc =} \ConstantTok{FALSE}\NormalTok{)}

\NormalTok{fit\_4\_groups }\OtherTok{\textless{}{-}} \FunctionTok{brm}\NormalTok{(formula,}
  \AttributeTok{family =} \FunctionTok{cumulative}\NormalTok{(}\StringTok{"logit"}\NormalTok{),}
  \AttributeTok{data =}\NormalTok{ d,}
  \AttributeTok{prior =}\NormalTok{ priors,}
  \AttributeTok{init =} \FunctionTok{rep}\NormalTok{(}\FunctionTok{list}\NormalTok{(inits), }\DecValTok{4}\NormalTok{),}
  \AttributeTok{chains =} \DecValTok{4}\NormalTok{, }\AttributeTok{iter =} \DecValTok{2000}\NormalTok{, }\AttributeTok{cores =} \DecValTok{4}\NormalTok{,}
  \AttributeTok{backend =} \StringTok{"cmdstanr"}\NormalTok{,}
  \AttributeTok{file =}\NormalTok{ here}\SpecialCharTok{::}\FunctionTok{here}\NormalTok{(}\StringTok{"models/fit\_4\_groups"}\NormalTok{),}
  \AttributeTok{save\_model =}\NormalTok{ here}\SpecialCharTok{::}\FunctionTok{here}\NormalTok{(}\StringTok{"stancode/fit\_4\_groups.stan"}\NormalTok{)}
\NormalTok{) }\SpecialCharTok{|\textgreater{}}
  \FunctionTok{add\_criterion}\NormalTok{(}\StringTok{"loo"}\NormalTok{)}
\end{Highlighting}
\end{Shaded}

\hypertarget{model-comparison-1}{%
\subsection{Model comparison}\label{model-comparison-1}}

\begin{Shaded}
\begin{Highlighting}[]
\FunctionTok{loo\_compare}\NormalTok{(}
\NormalTok{  fit\_2\_groups,}
\NormalTok{  fit\_3\_groups,}
\NormalTok{  fit\_3\_groups\_alt,}
\NormalTok{  fit\_4\_groups)}
\end{Highlighting}
\end{Shaded}

\begin{verbatim}
##                  elpd_diff se_diff
## fit_2_groups      0.0       0.0   
## fit_3_groups     -1.7       2.3   
## fit_3_groups_alt -2.0       2.5   
## fit_4_groups     -2.9       3.4
\end{verbatim}

\hypertarget{expectations-of-posterior-predictive-distribution}{%
\subsection{Expectations of posterior predictive distribution}\label{expectations-of-posterior-predictive-distribution}}

\begin{Shaded}
\begin{Highlighting}[]
\NormalTok{epred\_2\_groups }\OtherTok{\textless{}{-}}\NormalTok{ d }\SpecialCharTok{|\textgreater{}}
  \FunctionTok{data\_grid}\NormalTok{(group2, test, oldnew) }\SpecialCharTok{|\textgreater{}}
  \FunctionTok{add\_epred\_draws}\NormalTok{(fit\_2\_groups,}
                  \AttributeTok{category =} \StringTok{"rating"}\NormalTok{,}
                  \AttributeTok{dpar =} \ConstantTok{TRUE}\NormalTok{,}
                  \AttributeTok{re\_formula =} \SpecialCharTok{\textasciitilde{}}\NormalTok{ID,}
                  \AttributeTok{ndraws =} \DecValTok{500}\NormalTok{)}

\NormalTok{epred\_3\_groups }\OtherTok{\textless{}{-}}\NormalTok{ d }\SpecialCharTok{|\textgreater{}}
  \FunctionTok{data\_grid}\NormalTok{(group3, test, oldnew) }\SpecialCharTok{|\textgreater{}}
  \FunctionTok{add\_epred\_draws}\NormalTok{(fit\_3\_groups,}
                  \AttributeTok{category =} \StringTok{"rating"}\NormalTok{,}
                  \AttributeTok{dpar =} \ConstantTok{TRUE}\NormalTok{,}
                  \AttributeTok{re\_formula =} \SpecialCharTok{\textasciitilde{}}\NormalTok{ID,}
                  \AttributeTok{ndraws =} \DecValTok{500}\NormalTok{)}

\NormalTok{epred\_3\_groups\_alt }\OtherTok{\textless{}{-}}\NormalTok{ d }\SpecialCharTok{|\textgreater{}}
  \FunctionTok{data\_grid}\NormalTok{(group3alt, test, oldnew) }\SpecialCharTok{|\textgreater{}}
  \FunctionTok{add\_epred\_draws}\NormalTok{(fit\_3\_groups\_alt,}
                  \AttributeTok{category =} \StringTok{"rating"}\NormalTok{,}
                  \AttributeTok{dpar =} \ConstantTok{TRUE}\NormalTok{,}
                  \AttributeTok{re\_formula =} \SpecialCharTok{\textasciitilde{}}\NormalTok{ID,}
                  \AttributeTok{ndraws =} \DecValTok{500}\NormalTok{)}

\NormalTok{epred\_4\_groups }\OtherTok{\textless{}{-}}\NormalTok{ d }\SpecialCharTok{|\textgreater{}}
  \FunctionTok{data\_grid}\NormalTok{(group4, test, oldnew) }\SpecialCharTok{|\textgreater{}}
  \FunctionTok{add\_epred\_draws}\NormalTok{(fit\_4\_groups,}
                  \AttributeTok{category =} \StringTok{"rating"}\NormalTok{,}
                  \AttributeTok{dpar =} \ConstantTok{TRUE}\NormalTok{,}
                  \AttributeTok{re\_formula =} \SpecialCharTok{\textasciitilde{}}\NormalTok{ID,}
                  \AttributeTok{ndraws =} \DecValTok{500}\NormalTok{)}
\end{Highlighting}
\end{Shaded}

\begin{Shaded}
\begin{Highlighting}[]
\NormalTok{epred\_2\_groups }\SpecialCharTok{|\textgreater{}}
  \FunctionTok{ggplot}\NormalTok{(}\FunctionTok{aes}\NormalTok{(}\AttributeTok{x =}\NormalTok{ oldnew, }\AttributeTok{y =}\NormalTok{ .epred, }\AttributeTok{color =}\NormalTok{ rating)) }\SpecialCharTok{+}
  \FunctionTok{stat\_pointinterval}\NormalTok{(}\AttributeTok{position =} \FunctionTok{position\_dodge}\NormalTok{(}\AttributeTok{width =}\NormalTok{ .}\DecValTok{4}\NormalTok{)) }\SpecialCharTok{+}
  \FunctionTok{facet\_grid}\NormalTok{(test }\SpecialCharTok{\textasciitilde{}}\NormalTok{ group2) }\SpecialCharTok{+}
  \FunctionTok{scale\_size\_continuous}\NormalTok{(}\AttributeTok{guide =} \ConstantTok{FALSE}\NormalTok{) }\SpecialCharTok{+}
  \FunctionTok{scale\_y\_continuous}\NormalTok{(}\AttributeTok{limits =} \FunctionTok{c}\NormalTok{(}\DecValTok{0}\NormalTok{, }\DecValTok{1}\NormalTok{)) }\SpecialCharTok{+}
  \CommentTok{\# scale\_color\_brewer(palette = "RdYlBu")}
  \CommentTok{\# scale\_size\_continuous(guide = "none") +}
  \CommentTok{\# scale\_color\_manual(values = brewer.pal(6, "Blues")[{-}c(1)]) +}
  \FunctionTok{scale\_color\_viridis\_d}\NormalTok{(}\AttributeTok{begin =} \FloatTok{0.0}\NormalTok{, }\AttributeTok{end =} \FloatTok{0.8}\NormalTok{) }\SpecialCharTok{+}
  \FunctionTok{theme\_clean}\NormalTok{()}
\end{Highlighting}
\end{Shaded}

\includegraphics{supplementary_files/figure-latex/unnamed-chunk-44-1.pdf}

\begin{Shaded}
\begin{Highlighting}[]
\NormalTok{epred\_3\_groups }\SpecialCharTok{|\textgreater{}}
  \FunctionTok{ggplot}\NormalTok{(}\FunctionTok{aes}\NormalTok{(}\AttributeTok{x =}\NormalTok{ rating, }\AttributeTok{y =}\NormalTok{ .epred, }\AttributeTok{color =}\NormalTok{ oldnew)) }\SpecialCharTok{+}
  \FunctionTok{stat\_pointinterval}\NormalTok{(}\AttributeTok{position =} \FunctionTok{position\_dodge}\NormalTok{(}\AttributeTok{width =}\NormalTok{ .}\DecValTok{4}\NormalTok{)) }\SpecialCharTok{+}
  \CommentTok{\# geom\_line(aes(group = oldnew)) +}
  \FunctionTok{facet\_grid}\NormalTok{(test }\SpecialCharTok{\textasciitilde{}}\NormalTok{ group3) }\SpecialCharTok{+}
  \FunctionTok{scale\_size\_continuous}\NormalTok{(}\AttributeTok{guide =} \ConstantTok{FALSE}\NormalTok{) }\SpecialCharTok{+}
  \CommentTok{\# scale\_y\_continuous(limits = c(0, 1)) +}
  \FunctionTok{expand\_limits}\NormalTok{(}\AttributeTok{y =} \DecValTok{0}\NormalTok{) }\SpecialCharTok{+}
  \CommentTok{\# scale\_color\_brewer(palette = "RdYlBu")}
  \FunctionTok{scale\_color\_viridis\_d}\NormalTok{(}\AttributeTok{begin =} \FloatTok{0.0}\NormalTok{, }\AttributeTok{end =} \FloatTok{0.8}\NormalTok{) }\SpecialCharTok{+}
  \FunctionTok{theme\_clean}\NormalTok{()}
\end{Highlighting}
\end{Shaded}

\includegraphics{supplementary_files/figure-latex/unnamed-chunk-48-1.pdf}

\begin{Shaded}
\begin{Highlighting}[]
\NormalTok{epred\_3\_groups\_alt }\SpecialCharTok{|\textgreater{}}
  \FunctionTok{ggplot}\NormalTok{(}\FunctionTok{aes}\NormalTok{(}\AttributeTok{x =}\NormalTok{ rating, }\AttributeTok{y =}\NormalTok{ .epred, }\AttributeTok{color =}\NormalTok{ oldnew)) }\SpecialCharTok{+}
  \FunctionTok{stat\_pointinterval}\NormalTok{(}\AttributeTok{position =} \FunctionTok{position\_dodge}\NormalTok{(}\AttributeTok{width =}\NormalTok{ .}\DecValTok{4}\NormalTok{)) }\SpecialCharTok{+}
  \CommentTok{\# geom\_line(aes(group = oldnew)) +}
  \FunctionTok{facet\_grid}\NormalTok{(test }\SpecialCharTok{\textasciitilde{}}\NormalTok{ group3alt) }\SpecialCharTok{+}
  \FunctionTok{scale\_size\_continuous}\NormalTok{(}\AttributeTok{guide =} \ConstantTok{FALSE}\NormalTok{) }\SpecialCharTok{+}
  \FunctionTok{expand\_limits}\NormalTok{(}\AttributeTok{y =} \DecValTok{0}\NormalTok{) }\SpecialCharTok{+}  
  \CommentTok{\# scale\_color\_brewer(palette = "RdYlBu")}
  \FunctionTok{scale\_color\_viridis\_d}\NormalTok{(}\AttributeTok{begin =} \FloatTok{0.0}\NormalTok{, }\AttributeTok{end =} \FloatTok{0.8}\NormalTok{) }\SpecialCharTok{+}
  \FunctionTok{theme\_clean}\NormalTok{()}
\end{Highlighting}
\end{Shaded}

\includegraphics{supplementary_files/figure-latex/unnamed-chunk-49-1.pdf}

\begin{Shaded}
\begin{Highlighting}[]
\NormalTok{epred\_4\_groups }\SpecialCharTok{|\textgreater{}}
  \FunctionTok{ggplot}\NormalTok{(}\FunctionTok{aes}\NormalTok{(}\AttributeTok{x =}\NormalTok{ rating, }\AttributeTok{y =}\NormalTok{ .epred, }\AttributeTok{color =}\NormalTok{ oldnew)) }\SpecialCharTok{+}
  \FunctionTok{stat\_pointinterval}\NormalTok{(}\AttributeTok{position =} \FunctionTok{position\_dodge}\NormalTok{(}\AttributeTok{width =}\NormalTok{ .}\DecValTok{4}\NormalTok{)) }\SpecialCharTok{+}
  \CommentTok{\# geom\_line(aes(group = oldnew)) +}
  \FunctionTok{facet\_grid}\NormalTok{(test }\SpecialCharTok{\textasciitilde{}}\NormalTok{ group4) }\SpecialCharTok{+}
  \FunctionTok{scale\_size\_continuous}\NormalTok{(}\AttributeTok{guide =} \ConstantTok{FALSE}\NormalTok{) }\SpecialCharTok{+}
  \FunctionTok{expand\_limits}\NormalTok{(}\AttributeTok{y =} \DecValTok{0}\NormalTok{) }\SpecialCharTok{+}  
  \CommentTok{\# scale\_color\_brewer(palette = "RdYlBu")}
  \FunctionTok{scale\_color\_viridis\_d}\NormalTok{(}\AttributeTok{begin =} \FloatTok{0.0}\NormalTok{, }\AttributeTok{end =} \FloatTok{0.8}\NormalTok{) }\SpecialCharTok{+}
  \FunctionTok{theme\_clean}\NormalTok{()}
\end{Highlighting}
\end{Shaded}

\includegraphics{supplementary_files/figure-latex/unnamed-chunk-50-1.pdf}

\hypertarget{posterior-expectations}{%
\section{Posterior expectations}\label{posterior-expectations}}

The following plots show the exptected learning scores, computed from the models posterior predictive distribution. This plot is shown merely to demonstrate that the models output can be used to create a plot that is similar to the learning score computed from the data. This could serve as a posterior predictive check, i.e.~in order to show that the models predictions closely match the empirical data (the model predicts data that are similar to the actual data) according to some desired metric (in this case learning score).

\begin{Shaded}
\begin{Highlighting}[]
\NormalTok{posterior\_expectations }\OtherTok{\textless{}{-}}\NormalTok{ epred\_4\_groups }\SpecialCharTok{|\textgreater{}} 
  \FunctionTok{mutate}\NormalTok{(}\AttributeTok{product =} \FunctionTok{as.double}\NormalTok{(rating) }\SpecialCharTok{*}\NormalTok{ .epred) }\SpecialCharTok{|\textgreater{}} 
  \CommentTok{\# group and convert to the sum{-}score metric}
  \FunctionTok{group\_by}\NormalTok{(group4, test, oldnew, .draw) }\SpecialCharTok{|\textgreater{}} 
  \FunctionTok{summarise}\NormalTok{(}\AttributeTok{mean\_rating =} \FunctionTok{sum}\NormalTok{(product)) }\SpecialCharTok{|\textgreater{}} 
  \FunctionTok{pivot\_wider}\NormalTok{(}\AttributeTok{names\_from =}\NormalTok{ oldnew, }\AttributeTok{values\_from =}\NormalTok{ mean\_rating) }\SpecialCharTok{|\textgreater{}} 
  \FunctionTok{mutate}\NormalTok{(}\AttributeTok{score =}\NormalTok{ old }\SpecialCharTok{{-}}\NormalTok{ new)}
\end{Highlighting}
\end{Shaded}

\begin{Shaded}
\begin{Highlighting}[]
\NormalTok{posterior\_expectations }\SpecialCharTok{|\textgreater{}} 
  \CommentTok{\# summarize}
  \FunctionTok{group\_by}\NormalTok{(group4, test) }\SpecialCharTok{|\textgreater{}} 
  \FunctionTok{mean\_qi}\NormalTok{(score) }\SpecialCharTok{|\textgreater{}} 
  \FunctionTok{ggplot}\NormalTok{(}\FunctionTok{aes}\NormalTok{(}\AttributeTok{y =}\NormalTok{ score, }\AttributeTok{x =}\NormalTok{ test, }\AttributeTok{color =}\NormalTok{ group4, }\AttributeTok{ymin =}\NormalTok{ .lower, }\AttributeTok{ymax =}\NormalTok{ .upper)) }\SpecialCharTok{+}
  \FunctionTok{geom\_line}\NormalTok{(}\FunctionTok{aes}\NormalTok{(}\AttributeTok{group =}\NormalTok{ group4), }\AttributeTok{linewidth =} \FloatTok{1.5}\NormalTok{, }\AttributeTok{linetype =} \StringTok{"dashed"}\NormalTok{) }\SpecialCharTok{+}
  \FunctionTok{geom\_pointinterval}\NormalTok{() }\SpecialCharTok{+}
  \FunctionTok{scale\_color\_viridis\_d}\NormalTok{(}\AttributeTok{direction =} \SpecialCharTok{{-}}\DecValTok{1}\NormalTok{, }\AttributeTok{option =} \StringTok{"C"}\NormalTok{, }\AttributeTok{end =}\NormalTok{ .}\DecValTok{85}\NormalTok{) }\SpecialCharTok{+}
  \FunctionTok{ylab}\NormalTok{(}\StringTok{"Learning score"}\NormalTok{) }\SpecialCharTok{+}
  \FunctionTok{xlab}\NormalTok{(}\StringTok{"Test"}\NormalTok{) }\SpecialCharTok{+}
  \FunctionTok{ylim}\NormalTok{(}\DecValTok{0}\NormalTok{, }\DecValTok{3}\NormalTok{) }\SpecialCharTok{+}
  \FunctionTok{theme\_clean}\NormalTok{()}
\end{Highlighting}
\end{Shaded}

\includegraphics{supplementary_files/figure-latex/unnamed-chunk-53-1.pdf}

\begin{Shaded}
\begin{Highlighting}[]
  \CommentTok{\# facet\_grid(. \textasciitilde{} test)}
  \CommentTok{\# facet\_wrap(\textasciitilde{}test)}
\end{Highlighting}
\end{Shaded}

\begin{Shaded}
\begin{Highlighting}[]
\NormalTok{posterior\_expectations }\SpecialCharTok{|\textgreater{}} 
  \FunctionTok{ggplot}\NormalTok{(}\FunctionTok{aes}\NormalTok{(}\AttributeTok{x =}\NormalTok{ test, }\AttributeTok{y =}\NormalTok{ score)) }\SpecialCharTok{+}
  \FunctionTok{stat\_interval}\NormalTok{(}\AttributeTok{.width =} \FunctionTok{c}\NormalTok{(.}\DecValTok{50}\NormalTok{, .}\DecValTok{80}\NormalTok{, .}\DecValTok{95}\NormalTok{)) }\SpecialCharTok{+}
  \FunctionTok{facet\_wrap}\NormalTok{(}\SpecialCharTok{\textasciitilde{}}\NormalTok{group4) }\SpecialCharTok{+}
  \FunctionTok{scale\_color\_brewer}\NormalTok{(}\AttributeTok{palette =} \StringTok{"Blues"}\NormalTok{) }\SpecialCharTok{+}
  \FunctionTok{theme\_clean}\NormalTok{()}
\end{Highlighting}
\end{Shaded}

\includegraphics{supplementary_files/figure-latex/unnamed-chunk-54-1.pdf}

\hypertarget{simulate-future-data}{%
\section{Simulate future data}\label{simulate-future-data}}

This is (in my opinion) a very valuable part of the data analysis, because we can use our model to simulate unseen subjects from the 4 groups, according to the estimted random effects structure. Here, I show one simulated subject from the musician group (rel/abs pitch) and one simulated synaesthete.

\begin{Shaded}
\begin{Highlighting}[]
\NormalTok{newdata }\OtherTok{\textless{}{-}} \FunctionTok{expand\_grid}\NormalTok{(}\AttributeTok{group3alt =} \FunctionTok{c}\NormalTok{(}\StringTok{"syn"}\NormalTok{, }\StringTok{"musician"}\NormalTok{),}
                       \AttributeTok{oldnew =} \FunctionTok{c}\NormalTok{(}\StringTok{"new"}\NormalTok{, }\StringTok{"old"}\NormalTok{),}
                       \AttributeTok{test =} \FunctionTok{c}\NormalTok{(}\StringTok{"colors"}\NormalTok{, }\StringTok{"tones"}\NormalTok{, }\StringTok{"combined"}\NormalTok{), }
                       \AttributeTok{ID =} \ConstantTok{NA}\NormalTok{) }\SpecialCharTok{|\textgreater{}} 
  \FunctionTok{mutate}\NormalTok{(}\FunctionTok{across}\NormalTok{(}\FunctionTok{where}\NormalTok{(is\_character), as\_factor),}
         \AttributeTok{group3alt =} \FunctionTok{fct\_relevel}\NormalTok{(group3alt,}
                                 \StringTok{"musician"}\NormalTok{, }\StringTok{"syn"}\NormalTok{),}
         \AttributeTok{test =} \FunctionTok{fct\_relevel}\NormalTok{(test,}
                            \StringTok{"colors"}\NormalTok{, }\StringTok{"tones"}\NormalTok{, }\StringTok{"combined"}\NormalTok{))}

\NormalTok{preds }\OtherTok{\textless{}{-}}\NormalTok{ fit\_3\_groups\_alt }\SpecialCharTok{|\textgreater{}} 
  \FunctionTok{add\_epred\_draws}\NormalTok{(}\AttributeTok{newdata =}\NormalTok{ newdata,}
                  \AttributeTok{category =} \StringTok{"rating"}\NormalTok{,}
                  \AttributeTok{dpar =} \ConstantTok{TRUE}\NormalTok{,}
                  \AttributeTok{re\_formula =} \SpecialCharTok{\textasciitilde{}}\NormalTok{(}\DecValTok{1} \SpecialCharTok{|}\NormalTok{ ID) }\SpecialCharTok{+}\NormalTok{ (}\DecValTok{1} \SpecialCharTok{|}\NormalTok{ item),}
                  \AttributeTok{ndraws =} \DecValTok{500}\NormalTok{,}
                  \AttributeTok{allow\_new\_levels =} \ConstantTok{TRUE}\NormalTok{,}
                  \AttributeTok{sample\_new\_levels =} \StringTok{"gaussian"}\NormalTok{)}
\end{Highlighting}
\end{Shaded}

\begin{Shaded}
\begin{Highlighting}[]
\NormalTok{preds }\SpecialCharTok{|\textgreater{}} 
  \FunctionTok{ggplot}\NormalTok{(}\FunctionTok{aes}\NormalTok{(}\AttributeTok{x =}\NormalTok{ rating, }\AttributeTok{y =}\NormalTok{ .epred, }\AttributeTok{color =}\NormalTok{ oldnew)) }\SpecialCharTok{+}
  \FunctionTok{stat\_pointinterval}\NormalTok{(}\AttributeTok{position =} \FunctionTok{position\_dodge}\NormalTok{(}\AttributeTok{width =}\NormalTok{ .}\DecValTok{2}\NormalTok{)) }\SpecialCharTok{+}
  \FunctionTok{facet\_grid}\NormalTok{(test }\SpecialCharTok{\textasciitilde{}}\NormalTok{ group3alt, }\AttributeTok{labeller =}\NormalTok{ label\_both) }\SpecialCharTok{+}
  \FunctionTok{scale\_size\_continuous}\NormalTok{(}\AttributeTok{guide =} \ConstantTok{FALSE}\NormalTok{) }\SpecialCharTok{+}
  \FunctionTok{expand\_limits}\NormalTok{(}\AttributeTok{y =} \DecValTok{0}\NormalTok{) }\SpecialCharTok{+}
  \CommentTok{\# scale\_color\_okabe\_ito() +}
  \FunctionTok{scale\_color\_viridis\_d}\NormalTok{(}\AttributeTok{begin =} \FloatTok{0.0}\NormalTok{, }\AttributeTok{end =} \FloatTok{0.8}\NormalTok{) }\SpecialCharTok{+}
  \FunctionTok{labs}\NormalTok{(}\AttributeTok{x =} \StringTok{"Rating"}\NormalTok{, }\AttributeTok{y =} \StringTok{"Predicted probability"}\NormalTok{, }\AttributeTok{color =} \StringTok{"Triplet"}\NormalTok{) }\SpecialCharTok{+}
  \FunctionTok{theme\_clean}\NormalTok{() }\SpecialCharTok{+}
  \FunctionTok{theme}\NormalTok{(}\AttributeTok{legend.position =} \StringTok{"bottom"}\NormalTok{)}
\end{Highlighting}
\end{Shaded}

\includegraphics{supplementary_files/figure-latex/unnamed-chunk-56-1.pdf}

In the next figure, I show 10 simulated synaesthetes in the combined test, and contrast these with ten simulated musicians with either relative or absolute pitch.

\begin{Shaded}
\begin{Highlighting}[]
\NormalTok{newdata }\OtherTok{\textless{}{-}} \FunctionTok{expand\_grid}\NormalTok{(}\AttributeTok{group3alt =} \StringTok{"syn"}\NormalTok{,}
                       \AttributeTok{oldnew =} \FunctionTok{c}\NormalTok{(}\StringTok{"new"}\NormalTok{, }\StringTok{"old"}\NormalTok{),}
                       \AttributeTok{test =} \StringTok{"combined"}\NormalTok{, }
                       \AttributeTok{ID =} \FunctionTok{str\_c}\NormalTok{(}\StringTok{"Synaesthete "}\NormalTok{, }\DecValTok{1}\SpecialCharTok{:}\DecValTok{10}\NormalTok{)) }\SpecialCharTok{|\textgreater{}} 
  \FunctionTok{mutate}\NormalTok{(}\FunctionTok{across}\NormalTok{(}\FunctionTok{where}\NormalTok{(is\_character), as\_factor),}
         \AttributeTok{test =} \FunctionTok{fct\_relevel}\NormalTok{(test,}
                            \StringTok{"colors"}\NormalTok{, }\StringTok{"tones"}\NormalTok{, }\StringTok{"combined"}\NormalTok{))}
\end{Highlighting}
\end{Shaded}

\begin{verbatim}
## Warning: 2 unknown levels in `f`: colors and tones
\end{verbatim}

\begin{Shaded}
\begin{Highlighting}[]
\NormalTok{preds }\OtherTok{\textless{}{-}}\NormalTok{ fit\_3\_groups\_alt }\SpecialCharTok{|\textgreater{}} 
  \FunctionTok{add\_epred\_draws}\NormalTok{(}\AttributeTok{newdata =}\NormalTok{ newdata,}
                  \AttributeTok{category =} \StringTok{"rating"}\NormalTok{,}
                  \AttributeTok{dpar =} \ConstantTok{TRUE}\NormalTok{,}
                  \AttributeTok{re\_formula =} \SpecialCharTok{\textasciitilde{}}\NormalTok{(}\DecValTok{1} \SpecialCharTok{|}\NormalTok{ ID) }\SpecialCharTok{+}\NormalTok{ (}\DecValTok{1} \SpecialCharTok{|}\NormalTok{ item),}
                  \AttributeTok{ndraws =} \DecValTok{500}\NormalTok{,}
                  \AttributeTok{allow\_new\_levels =} \ConstantTok{TRUE}\NormalTok{,}
                  \AttributeTok{sample\_new\_levels =} \StringTok{"gaussian"}\NormalTok{)}
\end{Highlighting}
\end{Shaded}

\begin{Shaded}
\begin{Highlighting}[]
\NormalTok{p\_syn }\OtherTok{\textless{}{-}}\NormalTok{ preds }\SpecialCharTok{|\textgreater{}} 
  \FunctionTok{ggplot}\NormalTok{(}\FunctionTok{aes}\NormalTok{(}\AttributeTok{x =}\NormalTok{ rating, }\AttributeTok{y =}\NormalTok{ .epred, }\AttributeTok{color =}\NormalTok{ oldnew)) }\SpecialCharTok{+}
  \FunctionTok{stat\_pointinterval}\NormalTok{(}\AttributeTok{position =} \FunctionTok{position\_dodge}\NormalTok{(}\AttributeTok{width =}\NormalTok{ .}\DecValTok{2}\NormalTok{)) }\SpecialCharTok{+}
  \FunctionTok{facet\_wrap}\NormalTok{(}\SpecialCharTok{\textasciitilde{}}\NormalTok{ ID) }\SpecialCharTok{+}
  \FunctionTok{scale\_size\_continuous}\NormalTok{(}\AttributeTok{guide =} \ConstantTok{FALSE}\NormalTok{) }\SpecialCharTok{+}
  \FunctionTok{expand\_limits}\NormalTok{(}\AttributeTok{y =} \DecValTok{0}\NormalTok{) }\SpecialCharTok{+}
  \CommentTok{\# scale\_color\_okabe\_ito() +}
  \FunctionTok{scale\_color\_viridis\_d}\NormalTok{(}\AttributeTok{begin =} \FloatTok{0.0}\NormalTok{, }\AttributeTok{end =} \FloatTok{0.8}\NormalTok{) }\SpecialCharTok{+}
  \FunctionTok{labs}\NormalTok{(}\AttributeTok{x =} \StringTok{"Rating"}\NormalTok{, }\AttributeTok{y =} \StringTok{"Predicted probability"}\NormalTok{, }\AttributeTok{color =} \StringTok{"Triplet"}\NormalTok{) }\SpecialCharTok{+}
  \FunctionTok{theme\_clean}\NormalTok{() }\SpecialCharTok{+}
  \FunctionTok{theme}\NormalTok{(}\AttributeTok{legend.position =} \StringTok{"bottom"}\NormalTok{)}
\end{Highlighting}
\end{Shaded}

\begin{Shaded}
\begin{Highlighting}[]
\NormalTok{p\_syn}
\end{Highlighting}
\end{Shaded}

\includegraphics{supplementary_files/figure-latex/unnamed-chunk-59-1.pdf}

\begin{Shaded}
\begin{Highlighting}[]
\NormalTok{newdata }\OtherTok{\textless{}{-}} \FunctionTok{expand\_grid}\NormalTok{(}\AttributeTok{group3alt =} \StringTok{"musician"}\NormalTok{,}
                       \AttributeTok{oldnew =} \FunctionTok{c}\NormalTok{(}\StringTok{"new"}\NormalTok{, }\StringTok{"old"}\NormalTok{),}
                       \AttributeTok{test =} \StringTok{"combined"}\NormalTok{, }
                       \AttributeTok{ID =} \FunctionTok{str\_c}\NormalTok{(}\StringTok{"Musician "}\NormalTok{, }\DecValTok{1}\SpecialCharTok{:}\DecValTok{10}\NormalTok{)) }\SpecialCharTok{|\textgreater{}} 
  \FunctionTok{mutate}\NormalTok{(}\FunctionTok{across}\NormalTok{(}\FunctionTok{where}\NormalTok{(is\_character), as\_factor),}
         \AttributeTok{test =} \FunctionTok{fct\_relevel}\NormalTok{(test,}
                            \StringTok{"colors"}\NormalTok{, }\StringTok{"tones"}\NormalTok{, }\StringTok{"combined"}\NormalTok{))}
\end{Highlighting}
\end{Shaded}

\begin{verbatim}
## Warning: 2 unknown levels in `f`: colors and tones
\end{verbatim}

\begin{Shaded}
\begin{Highlighting}[]
\NormalTok{preds }\OtherTok{\textless{}{-}}\NormalTok{ fit\_3\_groups\_alt }\SpecialCharTok{|\textgreater{}} 
  \FunctionTok{add\_epred\_draws}\NormalTok{(}\AttributeTok{newdata =}\NormalTok{ newdata,}
                  \AttributeTok{category =} \StringTok{"rating"}\NormalTok{,}
                  \AttributeTok{dpar =} \ConstantTok{TRUE}\NormalTok{,}
                  \AttributeTok{re\_formula =} \SpecialCharTok{\textasciitilde{}}\NormalTok{(}\DecValTok{1} \SpecialCharTok{|}\NormalTok{ ID) }\SpecialCharTok{+}\NormalTok{ (}\DecValTok{1} \SpecialCharTok{|}\NormalTok{ item),}
                  \AttributeTok{ndraws =} \DecValTok{500}\NormalTok{,}
                  \AttributeTok{allow\_new\_levels =} \ConstantTok{TRUE}\NormalTok{,}
                  \AttributeTok{sample\_new\_levels =} \StringTok{"gaussian"}\NormalTok{)}
\end{Highlighting}
\end{Shaded}

\begin{Shaded}
\begin{Highlighting}[]
\NormalTok{p\_musician }\OtherTok{\textless{}{-}}\NormalTok{ preds }\SpecialCharTok{|\textgreater{}} 
  \FunctionTok{ggplot}\NormalTok{(}\FunctionTok{aes}\NormalTok{(}\AttributeTok{x =}\NormalTok{ rating, }\AttributeTok{y =}\NormalTok{ .epred, }\AttributeTok{color =}\NormalTok{ oldnew)) }\SpecialCharTok{+}
  \FunctionTok{stat\_pointinterval}\NormalTok{(}\AttributeTok{position =} \FunctionTok{position\_dodge}\NormalTok{(}\AttributeTok{width =}\NormalTok{ .}\DecValTok{2}\NormalTok{)) }\SpecialCharTok{+}
  \FunctionTok{facet\_wrap}\NormalTok{(}\SpecialCharTok{\textasciitilde{}}\NormalTok{ ID) }\SpecialCharTok{+}
  \FunctionTok{scale\_size\_continuous}\NormalTok{(}\AttributeTok{guide =} \ConstantTok{FALSE}\NormalTok{) }\SpecialCharTok{+}
  \FunctionTok{expand\_limits}\NormalTok{(}\AttributeTok{y =} \DecValTok{0}\NormalTok{) }\SpecialCharTok{+}
  \CommentTok{\# scale\_color\_okabe\_ito() +}
  \FunctionTok{scale\_color\_viridis\_d}\NormalTok{(}\AttributeTok{begin =} \FloatTok{0.0}\NormalTok{, }\AttributeTok{end =} \FloatTok{0.8}\NormalTok{) }\SpecialCharTok{+}
  \FunctionTok{labs}\NormalTok{(}\AttributeTok{x =} \StringTok{"Rating"}\NormalTok{, }\AttributeTok{y =} \StringTok{"Predicted probability"}\NormalTok{, }\AttributeTok{color =} \StringTok{"Triplet"}\NormalTok{) }\SpecialCharTok{+}
  \FunctionTok{theme\_clean}\NormalTok{() }\SpecialCharTok{+}
  \FunctionTok{theme}\NormalTok{(}\AttributeTok{legend.position =} \StringTok{"bottom"}\NormalTok{)}
\end{Highlighting}
\end{Shaded}

\begin{Shaded}
\begin{Highlighting}[]
\NormalTok{p\_musician}
\end{Highlighting}
\end{Shaded}

\includegraphics{supplementary_files/figure-latex/unnamed-chunk-62-1.pdf}

Overall, we can predict from these simulations that synaethetes mainly differ from musicians with relative and absolute pitch in their use of the extreme categories (1 and 5). When presented with previously unseen triplets, synaesthetes consistently use a rating of 1 to reject new items with great confidence, while musicians without synaesthesia are seemingly unable to identify previously unseen triplets with comparable confidence. The differences are less pronounced at the other end of the rating scale (identifying previoulsy seen triplets).

\newpage

\hypertarget{references}{%
\section{References}\label{references}}

\hypertarget{refs}{}
\begin{CSLReferences}{1}{0}
\leavevmode\vadjust pre{\hypertarget{ref-liddellAnalyzingOrdinalData2018a}{}}%
Liddell, T. M., \& Kruschke, J. K. (2018). Analyzing ordinal data with metric models: {What} could possibly go wrong? \emph{Journal of Experimental Social Psychology}, \emph{79}, 328--348. \url{https://doi.org/10.1016/j.jesp.2018.08.009}

\leavevmode\vadjust pre{\hypertarget{ref-moreyConfidenceIntervalsNormalized2008a}{}}%
Morey, R. D. (2008). Confidence {Intervals} from {Normalized Data}: {A} correction to {Cousineau} (2005). \emph{Tutorials in Quantitative Methods for Psychology}, \emph{4}(2), 61--64. \url{https://doi.org/10.20982/tqmp.04.2.p061}

\end{CSLReferences}


\end{document}
